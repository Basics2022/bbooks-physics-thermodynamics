%% Generated by Sphinx.
\def\sphinxdocclass{jupyterBook}
\documentclass[letterpaper,10pt,english]{jupyterBook}
\ifdefined\pdfpxdimen
   \let\sphinxpxdimen\pdfpxdimen\else\newdimen\sphinxpxdimen
\fi \sphinxpxdimen=.75bp\relax
\ifdefined\pdfimageresolution
    \pdfimageresolution= \numexpr \dimexpr1in\relax/\sphinxpxdimen\relax
\fi
%% let collapsible pdf bookmarks panel have high depth per default
\PassOptionsToPackage{bookmarksdepth=5}{hyperref}
%% turn off hyperref patch of \index as sphinx.xdy xindy module takes care of
%% suitable \hyperpage mark-up, working around hyperref-xindy incompatibility
\PassOptionsToPackage{hyperindex=false}{hyperref}
%% memoir class requires extra handling
\makeatletter\@ifclassloaded{memoir}
{\ifdefined\memhyperindexfalse\memhyperindexfalse\fi}{}\makeatother

\PassOptionsToPackage{warn}{textcomp}

\catcode`^^^^00a0\active\protected\def^^^^00a0{\leavevmode\nobreak\ }
\usepackage{cmap}
\usepackage{fontspec}
\defaultfontfeatures[\rmfamily,\sffamily,\ttfamily]{}
\usepackage{amsmath,amssymb,amstext}
\usepackage{polyglossia}
\setmainlanguage{english}



\setmainfont{FreeSerif}[
  Extension      = .otf,
  UprightFont    = *,
  ItalicFont     = *Italic,
  BoldFont       = *Bold,
  BoldItalicFont = *BoldItalic
]
\setsansfont{FreeSans}[
  Extension      = .otf,
  UprightFont    = *,
  ItalicFont     = *Oblique,
  BoldFont       = *Bold,
  BoldItalicFont = *BoldOblique,
]
\setmonofont{FreeMono}[
  Extension      = .otf,
  UprightFont    = *,
  ItalicFont     = *Oblique,
  BoldFont       = *Bold,
  BoldItalicFont = *BoldOblique,
]



\usepackage[Bjarne]{fncychap}
\usepackage[,numfigreset=1,mathnumfig]{sphinx}

\fvset{fontsize=\small}
\usepackage{geometry}


% Include hyperref last.
\usepackage{hyperref}
% Fix anchor placement for figures with captions.
\usepackage{hypcap}% it must be loaded after hyperref.
% Set up styles of URL: it should be placed after hyperref.
\urlstyle{same}

\addto\captionsenglish{\renewcommand{\contentsname}{Thermodynamics}}

\usepackage{sphinxmessages}



        % Start of preamble defined in sphinx-jupyterbook-latex %
         \usepackage[Latin,Greek]{ucharclasses}
        \usepackage{unicode-math}
        % fixing title of the toc
        \addto\captionsenglish{\renewcommand{\contentsname}{Contents}}
        \hypersetup{
            pdfencoding=auto,
            psdextra
        }
        % End of preamble defined in sphinx-jupyterbook-latex %
        

\title{Classical Thermodynamics}
\date{Jan 26, 2025}
\release{}
\author{basics}
\newcommand{\sphinxlogo}{\vbox{}}
\renewcommand{\releasename}{}
\makeindex
\begin{document}

\pagestyle{empty}
\sphinxmaketitle
\pagestyle{plain}
\sphinxtableofcontents
\pagestyle{normal}
\phantomsection\label{\detokenize{intro::doc}}


\sphinxAtStartPar
This material is part of the \sphinxhref{https://basics2022.github.io/bbooks}{\sphinxstylestrong{basics\sphinxhyphen{}books project}}. It is also available as a \DUrole{xref,download,myst}{.pdf document}.

\sphinxAtStartPar
Classical thermodynamics deal with energy, its forms and the ways it can be transferred.

\sphinxAtStartPar
Classical thermodynamics provides a macroscopic, average, description of complex systems composed by a “huge” number of elementary components, in equilibrium: equilibrium means statistical equilibrium, resulting in stationary macroscopic observable physical quatities \sphinxhyphen{} thermodynamic variables \sphinxhyphen{} even though non\sphinxhyphen{}stationary microscopic dynamics of the system. The development of thermodynamics is closely related with the development of chemistry and the re\sphinxhyphen{}born of an atomic theory of the matter. Classical thermodynamics is the results of studies on matter and energy performed from the early experiments of the XVII centuries on gas to the mathematical formulation of the late XIX century by Gibbs. A consistent theory of thermodynamics starts from conservation of mass and balance of total energy \sphinxhyphen{} whose changes result from input of work or heat \sphinxhyphen{}, provides a description of natural tendencies in nature, through Clausius’ statement of the second principle \sphinxhyphen{} degradation of mechanical energy and heat direction from hot to cold bodiess \sphinxhyphen{} in terms of a thermodynamic state variable, called entropy, and provides a definition of a thermodynamic scale of temperature \sphinxhyphen{} defined in terms of energy and entropy.



\sphinxstepscope


\part{Thermodynamics}

\sphinxstepscope




\chapter{Brief History of Thermodynamics}
\label{\detokenize{ch/history:brief-history-of-thermodynamics}}\label{\detokenize{ch/history:classical-thermodynamics-history}}\label{\detokenize{ch/history::doc}}
\sphinxAtStartPar
See introduction to thermodynamics in the material about \sphinxhref{https://basics2022.github.io/bbooks-physics-hs/intro.html}{physics for high\sphinxhyphen{}schools}, relying on your browser translation (Italian only, so far)
\begin{itemize}
\item {} 
\sphinxAtStartPar
\sphinxhref{https://basics2022.github.io/bbooks-physics-hs/ch/thermodynamics/foundation.html}{introduction to thermodynamics}

\item {} 
\sphinxAtStartPar
\sphinxhref{https://basics2022.github.io/bbooks-physics-hs/ch/thermodynamics/foundation-experiments.html}{brief history of experiments in thermodynamics}

\end{itemize}



\sphinxstepscope


\chapter{Principles of Thermodynamics}
\label{\detokenize{ch/principles:principles-of-thermodynamics}}\label{\detokenize{ch/principles:physics-hs-thermodynamics-principles}}\label{\detokenize{ch/principles::doc}}
\sphinxAtStartPar
In this chapter, the principles of classical thermodynamics, along with the relevant concepts and mathematical formalism, are introduced.
The principles of thermodynamics are presented for \sphinxstylestrong{closed systems}, and subsequently extended to open systems.


\begin{itemize}
\item {} 
\sphinxAtStartPar
The {\hyperref[\detokenize{ch/principles-lavoisier:physics-hs-thermodynamics-foundation-principles-lavoisier}]{\sphinxcrossref{\DUrole{std,std-ref}{\sphinxstylestrong{principle of conservation of mass \sphinxhyphen{} Lavoisier’s principle}}}}}, valid in classical mechanics and summarized by the formula \sphinxstyleemphasis{“nothing is created, nothing is destroyed, but everything is transformed”}, states that in a closed system, the mass is constant,
\begin{equation*}
\begin{split}d M = 0 \ .\end{split}
\end{equation*}
\item {} 
\sphinxAtStartPar
The {\hyperref[\detokenize{ch/principles-first:physics-hs-thermodynamics-foundation-principles-first}]{\sphinxcrossref{\DUrole{std,std-ref}{\sphinxstylestrong{first law of thermodynamics}}}}} provides the general form of the \sphinxstyleemphasis{total} energy balance of a closed system, recognizing the work done by external forces \(\delta L^{ext}\) and the heat \(\delta Q^{ext}\) exchanged between the system and the external environment as the causes of the variation of the total energy of the system.
\begin{equation*}
\begin{split}d E^{tot} = \delta L^{ext} + \delta Q^{ext} \ .\end{split}
\end{equation*}
\item {} 
\sphinxAtStartPar
The work of \sphinxstylestrong{Gibbs} provides {\hyperref[\detokenize{ch/principles-gibbs-phase-rule:physics-hs-thermodynamics-foundation-principles-gibbs-phase-rule}]{\sphinxcrossref{\DUrole{std,std-ref}{\sphinxstylestrong{the necessary concepts and a rigorous mathematical formalization}}}}} of classical thermodynamics. The concepts of internal energy, state variables, and Gibbs’ phase rule are introduced; additionally, some \DUrole{xref,myst}{phase diagrams} for representing the state of a system and thermodynamic transformations are presented, which will be used in subsequent chapters.

\item {} 
\sphinxAtStartPar
The {\hyperref[\detokenize{ch/principles-second:physics-hs-thermodynamics-foundation-principles-second}]{\sphinxcrossref{\DUrole{std,std-ref}{\sphinxstylestrong{second law of thermodynamics}}}}} describes natural tendencies: the dissipation of macroscopic mechanical energy and the transfer of heat from a hot body to a cold body, in a principle expressible in terms of \sphinxstylestrong{entropy},
\begin{equation*}
\begin{split}d S \ge \frac{\delta Q}{T} \ .\end{split}
\end{equation*}
\item {} 
\sphinxAtStartPar
Finally, the balance of physical quantities for {\hyperref[\detokenize{ch/principles-open:physics-hs-thermodynamics-foundation-principles-open}]{\sphinxcrossref{\DUrole{std,std-ref}{\sphinxstylestrong{open systems}}}}} is derived by modifying the balance equations for closed systems, introducing terms for the \sphinxstylestrong{flux of physical quantities due to the transport of matter} across the system’s boundary.

\end{itemize}



\sphinxstepscope


\section{Lavoisier’s Principle}
\label{\detokenize{ch/principles-lavoisier:lavoisier-s-principle}}\label{\detokenize{ch/principles-lavoisier:physics-hs-thermodynamics-foundation-principles-lavoisier}}\label{\detokenize{ch/principles-lavoisier::doc}}
\sphinxAtStartPar
In the context of classical mechanics, the principle of mass conservation—also known as Lavoisier’s principle—states that the mass \(M\) of a closed system is constant,
\begin{equation*}
\begin{split}d M = 0 \ ,\end{split}
\end{equation*}
\sphinxAtStartPar
that is, \sphinxstyleemphasis{“nothing is created, nothing is destroyed, everything is transformed.”}
\begin{itemize}
\item {} 
\sphinxAtStartPar
This principle was discovered by early chemists through the measurement of the mass of products and reactants in chemical reaction experiments.

\item {} 
\sphinxAtStartPar
This principle ceases to hold in the framework of Einstein’s theory of relativity, which recognizes the equivalence of mass and energy: mass and energy are two representations of a single physical quantity and are part of a balance equation. In the special case of a body at rest, this reduces to the famous expression \(E = m c^2\).

\end{itemize}

\sphinxstepscope


\section{First Law of Thermodynamics}
\label{\detokenize{ch/principles-first:first-law-of-thermodynamics}}\label{\detokenize{ch/principles-first:physics-hs-thermodynamics-foundation-principles-first}}\label{\detokenize{ch/principles-first::doc}}
\sphinxAtStartPar
The first law of thermodynamics is the total energy balance for closed systems. The variation in total energy \(d E^{tot}\) of a closed system is due to the work \(\delta L^{ext}\) performed on the system by external macroscopic actions and the heat \(\delta Q^{ext}\) transferred to the system from the outside,
\begin{equation*}
\begin{split}d E^{tot} = \delta L^{ext} + \delta Q^{ext} \ .\end{split}
\end{equation*}
\sphinxAtStartPar
Classical thermodynamics provides an average macroscopic description of the microscopic dynamics of a large number of elementary components (\sphinxstylestrong{todo} \sphinxstyleemphasis{atomic theory}). The total energy of the system can thus be interpreted as the sum of a macroscopic kinetic contribution and a microscopic content, both kinetic and potential; heat can be interpreted as the work performed on the system by microscopic actions.



\sphinxstepscope


\section{Gibbs: Internal Energy, Phase Rule, and Multivariable Functions}
\label{\detokenize{ch/principles-gibbs-phase-rule:gibbs-internal-energy-phase-rule-and-multivariable-functions}}\label{\detokenize{ch/principles-gibbs-phase-rule:physics-hs-thermodynamics-foundation-principles-gibbs-phase-rule}}\label{\detokenize{ch/principles-gibbs-phase-rule::doc}}
\sphinxAtStartPar
Following Gibbs’ work, this section introduces concepts such as {\hyperref[\detokenize{ch/principles-gibbs-phase-rule:physics-hs-thermodynamics-foundation-principles-gibbs-phase-rule-state-vars}]{\sphinxcrossref{\DUrole{std,std-ref}{state variables}}}} and {\hyperref[\detokenize{ch/principles-gibbs-phase-rule:physics-hs-thermodynamics-foundation-principles-gibbs-phase-rule-internal-energy}]{\sphinxcrossref{\DUrole{std,std-ref}{internal energy}}}}, as well as the {\hyperref[\detokenize{ch/principles-gibbs-phase-rule:physics-hs-thermodynamics-foundation-principles-gibbs-phase-rule-gibbs-phase-rule}]{\sphinxcrossref{\DUrole{std,std-ref}{Gibbs phase rule}}}}. Later, the {\hyperref[\detokenize{ch/principles-gibbs-phase-rule:physics-hs-thermodynamics-foundation-principles-gibbs-phase-rule-first}]{\sphinxcrossref{\DUrole{std,std-ref}{first law of thermodynamics is reformulated}}}} using the formalism introduced by Gibbs, which allows identifying the state of a system with a limited number of independent state variables and expressing the other (dependent) state variables as functions of multiple variables.


\subsection{State Variables}
\label{\detokenize{ch/principles-gibbs-phase-rule:state-variables}}\label{\detokenize{ch/principles-gibbs-phase-rule:physics-hs-thermodynamics-foundation-principles-gibbs-phase-rule-state-vars}}\label{ch/principles-gibbs-phase-rule:definition-0}
\begin{sphinxadmonition}{note}{Definition 1 (State Variable)}



\sphinxAtStartPar
A state variable of a system is a physical property of the system that depends exclusively on the current state of the system.
\end{sphinxadmonition}
\label{ch/principles-gibbs-phase-rule:example-1}
\begin{sphinxadmonition}{note}{Example 1 (State Variables and Non\sphinxhyphen{}State Variables)}



\sphinxAtStartPar
State variables include temperature, pressure, internal energy, entropy, etc.
Non\sphinxhyphen{}state variables include work or heat exchanged by the system. \sphinxstylestrong{todo}
\end{sphinxadmonition}


\subsection{Internal Energy}
\label{\detokenize{ch/principles-gibbs-phase-rule:internal-energy}}\label{\detokenize{ch/principles-gibbs-phase-rule:physics-hs-thermodynamics-foundation-principles-gibbs-phase-rule-internal-energy}}\label{ch/principles-gibbs-phase-rule:definition-2}
\begin{sphinxadmonition}{note}{Definition 2 (Internal Energy)}



\sphinxAtStartPar
The internal energy of a system is defined as the difference between the total energy and the macroscopic kinetic energy of the system,
\begin{equation*}
\begin{split}E = E^{tot} - K \ .\end{split}
\end{equation*}\end{sphinxadmonition}

\sphinxAtStartPar
It is possible to derive a balance for the internal energy of a closed system by subtracting the balance of kinetic energy described by the kinetic energy theorem from the balance of total energy provided by the first law of thermodynamics,
\begin{equation*}
\begin{split}\begin{aligned}
  d E^{tot} & = \delta L^{ext} + \delta Q^{ext} \\
  d K       & = \delta L^{ext} + \delta L^{int} \ ,
\end{aligned}\end{split}
\end{equation*}
\sphinxAtStartPar
The energy balance for the internal energy of a closed system then becomes
\begin{equation*}
\begin{split}d E = \delta Q^{ext} - \delta L^{int} \ .\end{split}
\end{equation*}

\subsection{Gibbs Phase Rule}
\label{\detokenize{ch/principles-gibbs-phase-rule:gibbs-phase-rule}}\label{\detokenize{ch/principles-gibbs-phase-rule:physics-hs-thermodynamics-foundation-principles-gibbs-phase-rule-gibbs-phase-rule}}\label{ch/principles-gibbs-phase-rule:definition-3}
\begin{sphinxadmonition}{note}{Definition 3 (Phase)}



\sphinxAtStartPar
A phase is defined as a portion of a chemical\sphinxhyphen{}physical system characterized by uniform chemical\sphinxhyphen{}physical (macroscopic) properties.
\end{sphinxadmonition}

\sphinxAtStartPar
\sphinxstylestrong{todo}
\begin{itemize}
\item {} 
\sphinxAtStartPar
Discussion of properties

\item {} 
\sphinxAtStartPar
Examples: a mixture of miscible gases constitutes a single phase, in which the individual components cannot be macroscopically distinguished; immiscible liquids remain macroscopically separated and thus constitute multiple phases, in which distinct chemical compositions can be macroscopically identified;…

\end{itemize}
\label{ch/principles-gibbs-phase-rule:proposition-4}
\begin{sphinxadmonition}{note}{Proposition 1 (Gibbs Phase Rule)}



\sphinxAtStartPar
The thermodynamic (equilibrium) state of a system is identified by a number \(F\) of independent \sphinxstylestrong{intensive} state variables, determined by the \sphinxstylestrong{Gibbs phase rule},
\begin{equation*}
\begin{split}F = C - P + 1 + W \ ,\end{split}
\end{equation*}
\sphinxAtStartPar
i.e., the number of independent intensive variables (or degrees of freedom), \(F\), of a system is a function of the number of independent components \(C\), the number of phases \(P\), and the number \(W\) of ways the system can manifest internal work, such as:
\begin{itemize}
\item {} 
\sphinxAtStartPar
internal mechanical stresses

\item {} 
\sphinxAtStartPar
contribution of surface tension

\item {} 
\sphinxAtStartPar
bond energy of the molecules of the components

\item {} 
\sphinxAtStartPar
contribution of the electromagnetic field

\end{itemize}
\end{sphinxadmonition}
\subsubsection*{Discussion of the Gibbs Phase Rule}

\sphinxAtStartPar
The equilibrium state of a system is defined by the values of the state variables, which for a non\sphinxhyphen{}electrically charged gas system are: temperature \(T\), pressure \(p\), and concentrations \(C_{c,\phi}\) of the individual components \(c=1:C\) in the individual phases \(\phi = 1:P\) within the system.

\sphinxAtStartPar
The state of the system is thus determined by the values of the \(1+W\) intensive thermodynamic variables, here \(W+1=2\) (\(T\), \(p\)), and the \(C \, P\) molar or mass fractions \(n_{c,\phi}\), for a total of \(N \, P + W + 1\) variables.
In general, these variables are constrained by some conditions:
\begin{itemize}
\item {} 
\sphinxAtStartPar
\(C \, (P-1)\) phase equilibrium conditions for each component, described by the equality of chemical potentials
\begin{equation*}
\begin{split}\mu_{c,\phi_1}(T,p) = \mu_{c,\phi_2}(T,p) = \dots = \mu_{c,\phi_P}(T,p)\end{split}
\end{equation*}
\item {} 
\sphinxAtStartPar
\(P\) unitarity conditions of the fractions
\begin{equation*}
\begin{split}\sum_{c} n_{c,\phi} = 1\end{split}
\end{equation*}
\end{itemize}

\sphinxAtStartPar
Thus, with \(C \, P + W + 1\) variables and \(P + C\, (P-1) = C \, P - C + P\) equations, we find that the problem can be determined by
\begin{equation*}
\begin{split}C \, P + W + 1 - C \, P + C - P  =  C - P + W + 1 = F \ ,\end{split}
\end{equation*}
\sphinxAtStartPar
independent variables.

\sphinxAtStartPar
\sphinxstylestrong{todo}
\begin{itemize}
\item {} 
\sphinxAtStartPar
Provide examples that clarify the definition of phase (e.g., pure solids or liquids represent phases on their own), and of independent component (e.g., chemical reactions, with no excess components, create constraints that reduce the number of independent substances, thanks to stoichiometric relationships between substances)

\item {} 
\sphinxAtStartPar
Discuss the role of phase fractions of a single component and the fact that they are not state variables; example phase transition from liquid to vapor: equilibrium is determined by the value of \(P\) (or \(T\)), and the vapor fraction is a result of other extensive variables of the system.

\end{itemize}
\label{ch/principles-gibbs-phase-rule:example-5}
\begin{sphinxadmonition}{note}{Example 2 (Closed system containing a single\sphinxhyphen{}component (or non\sphinxhyphen{}reactive), single\sphinxhyphen{}phase, electrically neutral (or not subjected to electromagnetic field))}



\sphinxAtStartPar
In a system consisting of a compressible gas, single\sphinxhyphen{}component and single\sphinxhyphen{}phase (gas phase), electrically neutral, \sphinxstylestrong{todo} \sphinxstyleemphasis{other?}, the only form of internal work is related to compression, \(\delta L^{int,rev} = P dV\), and thus \(W = 1\). Therefore, the system requires
\begin{equation*}
\begin{split}F = C - P + 1 + W = 1 - 1 + 1 + 1 = 2 \ ,\end{split}
\end{equation*}
\sphinxAtStartPar
state variables to define the system state.
\end{sphinxadmonition}
\label{ch/principles-gibbs-phase-rule:example-6}
\begin{sphinxadmonition}{note}{Example 3 (Open system containing a single\sphinxhyphen{}component (or non\sphinxhyphen{}reactive), single\sphinxhyphen{}phase, electrically neutral (or not subjected to electromagnetic field))}



\sphinxAtStartPar
In an open system, the variation in energy of the system also depends on the variation in the amount of gas contained in it. Thus, in general, there are \(W = 2\) ways to change the system’s energy: by compression work or by a flow of matter into the system. Therefore, \(F=3\) state variables are needed to define the system’s state.
\end{sphinxadmonition}
\label{ch/principles-gibbs-phase-rule:example-7}
\begin{sphinxadmonition}{note}{Example 4 (Reactive gas mixture in a closed system)}



\sphinxAtStartPar
In a reactive gas mixture consisting of two compounds \(A\), \(B\) in equilibrium according to the equilibrium reaction
\begin{equation*}
\begin{split}n_a A + n_b B \leftrightarrow n A_a B_b\end{split}
\end{equation*}
\sphinxAtStartPar
the energy of the system depends on the mechanical compression work of the gas mixture and the quantities of the 3 compounds present in the gas. However, the variation of these compounds is not independent but determined by the equilibrium reaction. Specifically,
\begin{equation*}
\begin{split}\begin{aligned}
  d n_{B} & = d n_{A} \frac{n_b}{n_a} \\
  d n_{A_a B_b} & = - d n_{A} \frac{n}{n_a} \\
\end{aligned}\end{split}
\end{equation*}
\sphinxAtStartPar
The reaction is thus determined by a single parameter. The variation in energy of the system is thus determined by \(W=2\) processes: from the compression work done on the system, and from the state of the reaction. To define the state of the system, \(F=3\) independent state variables are needed, such as \sphinxstylestrong{todo} \(T\), \(P\), \(n_A\)? Are the chemical potentials \(\mu_A\), \(\mu_B\), \(\mu_{A_a B_b}\) required? Are they uniquely determined?
\end{sphinxadmonition}
\label{ch/principles-gibbs-phase-rule:example-8}
\begin{sphinxadmonition}{note}{Example 5 (Single\sphinxhyphen{}component system during a phase transition)}



\sphinxAtStartPar
\sphinxstylestrong{First\sphinxhyphen{}order phase transition.} During a first\sphinxhyphen{}order phase transition, two phases, \(P = 2\), are simultaneously present in the system. According to the Gibbs phase rule, the state of the system is determined by
\begin{equation*}
\begin{split}F = C - P + 1 + W = 1 - 2 + 1 + 1 = 1 \ ,\end{split}
\end{equation*}
\sphinxAtStartPar
state variable.

\sphinxAtStartPar
\sphinxstylestrong{Critical point.} The critical point in the phase diagram of a single\sphinxhyphen{}component system defines the condition where three phases, \(P = 3\), are simultaneously present in the system. According to the Gibbs phase rule, the state of the system is determined by \(F = 0\) state variables: the state of the system at the critical point is uniquely defined, with no degrees of freedom.
\end{sphinxadmonition}
\label{ch/principles-gibbs-phase-rule:example-9}
\begin{sphinxadmonition}{note}{Example 6 (Solid)}



\sphinxAtStartPar
In the absence of other physical phenomena, the only form of work in a solid is related to deformation work, \(\delta L^{int,rev} = \sigma_{ij} \, d \varepsilon_{ij} dV\). The strain tensor is second\sphinxhyphen{}order and symmetric, and therefore has 6 independent components in 3\sphinxhyphen{}dimensional space, so \(W=6\)
\end{sphinxadmonition}
\label{ch/principles-gibbs-phase-rule:example-10}
\begin{sphinxadmonition}{note}{Example 7 (Solid mixtures)}


\begin{itemize}
\item {} 
\sphinxAtStartPar
Phases in solid mixtures \sphinxstylestrong{todo}

\end{itemize}
\end{sphinxadmonition}
\label{ch/principles-gibbs-phase-rule:example-11}
\begin{sphinxadmonition}{note}{Example 8 (Influence of the electromagnetic field)}


\begin{itemize}
\item {} 
\sphinxAtStartPar
Electric field and magnetization \sphinxstylestrong{todo}

\end{itemize}
\end{sphinxadmonition}


\subsection{First Law in terms of state variables}
\label{\detokenize{ch/principles-gibbs-phase-rule:first-law-in-terms-of-state-variables}}\label{\detokenize{ch/principles-gibbs-phase-rule:physics-hs-thermodynamics-foundation-principles-gibbs-phase-rule-first}}
\sphinxAtStartPar
Internal energy is an extensive variable of a thermodynamic system. In general, it can be written as a function of extensive variables that represent the ways the system manifests its internal energy (\sphinxstylestrong{todo} \sphinxstyleemphasis{due to work done on it, heat transferred to the system, and its chemical composition, thus the energy contained in bonds}).
\begin{equation*}
\begin{split}E(S, X_k)\end{split}
\end{equation*}
\sphinxAtStartPar
where \( X_k \) are the state variables whose variations are associated with reversible internal work, and \( S \) is the state variable whose variation is associated with heat exchange with the external environment and internal dissipative actions. \sphinxstylestrong{todo} \sphinxstyleemphasis{referring to the chapter on functions and multivariable calculus}

\sphinxAtStartPar
Assuming the function \( E \) is continuous and differentiable, at least piecewise, the exact differential of internal energy can be written as a function of the increments of the independent variables:
\begin{equation*}
\begin{split}
dE = \left. \dfrac{\partial E}{\partial S} \right|_{\mathbf{X}} d S + \left. \dfrac{\partial E}{\partial X_k} \right|_{S} d X_k = 
T \, d S + \sum_k F_k \, d X_k
\end{split}
\end{equation*}
\sphinxAtStartPar
where \( F_k \) are the generalized forces associated with changes in the generalized coordinates \( X_k \), and \( T \) and \( S \) correspond to temperature and entropy as discussed further below. \sphinxstylestrong{todo}.

\begin{sphinxadmonition}{note}{Note:}
\sphinxAtStartPar
As shown later, with this formalism, it is straightforward to express {\hyperref[\detokenize{ch/principles-gibbs-phase-rule:physics-hs-thermodynamics-foundation-principles-gibbs-phase-rule-second}]{\sphinxcrossref{\DUrole{std,std-ref}{the \sphinxstylestrong{second} and \sphinxstylestrong{third} laws of thermodynamics}}}} as:
\begin{itemize}
\item {} 
\sphinxAtStartPar
\( dS \ge \dfrac{\delta Q^{ext}}{T} \)

\item {} 
\sphinxAtStartPar
\( T \ge 0 \)

\end{itemize}
\end{sphinxadmonition}

\sphinxAtStartPar
The expression for the differential of internal energy can be compared with the energy balance written in terms of heat transferred to the system and internal work:
\begin{equation*}
\begin{split}
dE = \delta Q^{ext} - \delta L^{int} = 
\delta Q^{ext} + \delta^+ D - \delta L^{int,rev}
\end{split}
\end{equation*}
\sphinxAtStartPar
where the internal work \( \delta L^{int} \) is recognized as the sum of a reversible contribution and a dissipative contribution, which is never negative: \( \delta L^{int} = \delta L^{int,rev} - \delta^+ D \).

\sphinxAtStartPar
Since \( dE \) is an exact differential and \( \delta L^{int,rev} \) is a reversible contribution, it follows that the sum of the two non\sphinxhyphen{}reversible contributions, \( \delta U := \delta Q^{ext} + \delta^+ D \), is a reversible contribution. Comparing the two expressions for the differential of internal energy, we can associate reversible internal work with the sum of works formed as the product of generalized forces \( F_k \) and changes in the state variables \( X_k \), and the term \( \delta U \) with the product \( T \, dS \):
\begin{equation*}
\begin{split}
\begin{cases}
  -\delta L^{int,rev} &= \sum_k F_k \, d X_k \\
  \delta U &= T \, dS
\end{cases}
\end{split}
\end{equation*}\subsubsection*{Temperature, \protect\( T \protect\), and entropy, \protect\( S \protect\)}

\sphinxAtStartPar
In the absence of external work done on the system, and in the absence of dissipation \( \delta^+ D = 0 \), we get:
\begin{equation*}
\begin{split}
dE^{tot} = dE = \delta Q^{ext}
\end{split}
\end{equation*}\begin{equation*}
\begin{split}
dS = \frac{\delta Q^{ext}}{T}
\end{split}
\end{equation*}
\sphinxAtStartPar
Consider a closed and isolated system made of two subsystems in equilibrium, which can exchange heat but not work.

\sphinxAtStartPar
The total energy of the system is constant, \( E = E_1 + E_2 \). If the two subsystems are not initially at the same temperature, we observe a heat flow from the hotter system to the colder one, which satisfies the inequality:
\begin{equation*}
\begin{split}
\frac{\delta Q_{12}}{T_1} + \frac{\delta Q_{21}}{T_2} \ge 0 \quad \rightarrow \quad dS_1 + dS_2 \ge 0
\end{split}
\end{equation*}
\sphinxAtStartPar
The quantity \( S = S_1 + S_2 \) is non\sphinxhyphen{}decreasing.




\subsection{Second and Third Laws of Thermodynamics}
\label{\detokenize{ch/principles-gibbs-phase-rule:second-and-third-laws-of-thermodynamics}}\label{\detokenize{ch/principles-gibbs-phase-rule:physics-hs-thermodynamics-foundation-principles-gibbs-phase-rule-second}}
\sphinxAtStartPar
The formalism introduced in this section allows for a rather natural formulation of the second law and a version of the third law of thermodynamics.

\sphinxAtStartPar
This formulation of the third law of thermodynamics states that the thermodynamic temperature is always positive:
\begin{equation*}
\begin{split}
T := \left.\frac{\partial E}{\partial S}\right|_{\mathbf{X}} > 0
\end{split}
\end{equation*}
\sphinxAtStartPar
\sphinxstylestrong{todo} \sphinxstyleemphasis{Add some words about the significance, in terms of molecular agitation and probability}

\sphinxAtStartPar
In cases where this form or consequence of the third law of thermodynamics holds, the second law of thermodynamics is a consequence of the non\sphinxhyphen{}negativity of dissipation and the heat transfer mechanism, as will be discussed in more detail in the discussion of {\hyperref[\detokenize{ch/principles-second:physics-hs-thermodynamics-principles-second-composite}]{\sphinxcrossref{\DUrole{std,std-ref}{composite systems}}}}.

\sphinxAtStartPar
In the general case of a \sphinxstylestrong{simple system}, using the definition \( dS = \frac{\delta U}{T} = \frac{\delta Q^{ext} + \delta^+ D}{T} \), the non\sphinxhyphen{}negativity of dissipation, \( \delta^+ D \ge 0 \), implies:
\begin{equation*}
\begin{split}
dS \ge \frac{\delta Q^{ext}}{T}
\end{split}
\end{equation*}
\sphinxAtStartPar
This is an expression of the {\hyperref[\detokenize{ch/principles-second:physics-hs-thermodynamics-foundation-principles-second}]{\sphinxcrossref{\DUrole{std,std-ref}{Clausius statement of the second law of thermodynamics}}}}.

\sphinxAtStartPar
\sphinxstylestrong{todo} \sphinxstylestrong{oss} \sphinxstyleemphasis{The third law of thermodynamics: 1. It seems not to be a fundamental principle; 2. For some systems with limited energy, the definition of temperature \( T := \left( \frac{\partial E}{\partial S} \right) \) produces a negative temperature} \sphinxstylestrong{todo} \sphinxstyleemphasis{add a section on statistical mechanics?}


\bigskip\hrule\bigskip


\sphinxAtStartPar
From L.E. Reichl, \sphinxstyleemphasis{A Modern Course in Statistical Physics}, with some inconsistencies \sphinxstylestrong{todo} \sphinxstyleemphasis{check!}
\begin{equation*}
\begin{split}
\delta W = - P dV + J dL + \sigma d A + V \left( \vec{e} \cdot d \vec{p} + \vec{h} \cdot d \vec{m}\right) + \phi d q
\end{split}
\end{equation*}\begin{itemize}
\item {} 
\sphinxAtStartPar
where \( J \), \( \sigma \) are tensions per unit length and area, \( d L \), \( d A \) are changes in length or area,

\item {} 
\sphinxAtStartPar
where \( \vec{e} \), \( \vec{h} \) are the electric and magnetic fields, \( \vec{p} \), \( \vec{m} \) are polarization and magnetization,

\item {} 
\sphinxAtStartPar
\( \phi \) is the electric potential, and \( q \) is electric charge (for open systems, otherwise \( dq \equiv 0 \) or net charge would be created!)

\end{itemize}
\label{ch/principles-gibbs-phase-rule:example-12}
\begin{sphinxadmonition}{note}{Example 9 (Closed Monocomponent Gas System)}



\sphinxAtStartPar
The energy of the system, \( E(S,V) \)
\begin{equation*}
\begin{split}
dE = T \, dS - P \, dV
\end{split}
\end{equation*}\end{sphinxadmonition}
\label{ch/principles-gibbs-phase-rule:example-13}
\begin{sphinxadmonition}{note}{Example 10 (Open Monocomponent Gas System)}



\sphinxAtStartPar
The energy of the system, \( E(S,V,N) \)
\begin{equation*}
\begin{split}
dE = T \, dS - P \, dV + \mu \, dN
\end{split}
\end{equation*}\end{sphinxadmonition}
\label{ch/principles-gibbs-phase-rule:example-14}
\begin{sphinxadmonition}{note}{Example 11 (Reactive Gas Mixture in a Closed System)}



\sphinxAtStartPar
The energy of the system
\begin{equation*}
\begin{split}
\begin{aligned}
  dE & = T \, dS - P \, dV + \mu_k \, dN_k = \\
     & = T \, dS - P \, dV + \left( \mu_k n_k \right) dN
\end{aligned}
\end{split}
\end{equation*}
\sphinxAtStartPar
where \( n_k \) are the stoichiometric coefficients (with sign) of the reaction, and \( N \) is a quantity that identifies the equilibrium of the reaction, such that the variation of each component can be written as \( dN_k = n_k \, dN \).
\end{sphinxadmonition}
\label{ch/principles-gibbs-phase-rule:example-15}
\begin{sphinxadmonition}{note}{Example 12 (Monocomponent Mixture During a Phase Transition)}



\sphinxAtStartPar
\sphinxstylestrong{todo} how to treat the phase fractions?
\end{sphinxadmonition}
\label{ch/principles-gibbs-phase-rule:example-16}
\begin{sphinxadmonition}{note}{Example 13 (Solid)}



\sphinxAtStartPar
Solid system with initial volume \( V \) under uniform stress and strain, with small deformations
\begin{equation*}
\begin{split}
dE = T \, dS - V \, \sigma_{ij} \, d \varepsilon_{ij}
\end{split}
\end{equation*}\end{sphinxadmonition}
\label{ch/principles-gibbs-phase-rule:example-17}
\begin{sphinxadmonition}{note}{Example 14 (Solid Mixtures)}



\sphinxAtStartPar
\sphinxstylestrong{todo}
\end{sphinxadmonition}
\label{ch/principles-gibbs-phase-rule:example-18}
\begin{sphinxadmonition}{note}{Example 15 (Influence of the Magnetic Field)}


\begin{equation*}
\begin{split}
dE = T \, dS - P \, d V + H \, dM
\end{split}
\end{equation*}
\sphinxAtStartPar
\sphinxstylestrong{todo}
\end{sphinxadmonition}

\sphinxstepscope


\section{Second Law of Thermodynamics \sphinxhyphen{} Clausius Statement}
\label{\detokenize{ch/principles-second:second-law-of-thermodynamics-clausius-statement}}\label{\detokenize{ch/principles-second:physics-hs-thermodynamics-foundation-principles-second}}\label{\detokenize{ch/principles-second::doc}}
\sphinxAtStartPar
The Clausius statement of the second law of thermodynamics can be formulated quite naturally using the formalism introduced. There are two other famous statements of the second law of thermodynamics, the Planck and Kelvin statements, which will be presented in the context of heat engines.


\subsection{Simple Systems}
\label{\detokenize{ch/principles-second:simple-systems}}\label{\detokenize{ch/principles-second:physics-hs-thermodynamics-foundation-principles-second-simple}}
\sphinxAtStartPar
The elementary variation of entropy \(d S\) of a simple closed system at uniform temperature \(T\) is greater than or equal to the ratio of the elementary heat flux introduced into the system and the temperature of the system itself,
\begin{equation*}
\begin{split}dS = \underbrace{\dfrac{\delta^+ D}{T}}_{\ge 0} + \dfrac{\delta Q^{ext}}{T} \ge \dfrac{\delta Q^{ext}}{T} \ .\end{split}
\end{equation*}
\sphinxAtStartPar
This is the Clausius statement of the second law of thermodynamics for simple systems with uniform temperature.


\subsection{Composite Discrete Systems}
\label{\detokenize{ch/principles-second:composite-discrete-systems}}\label{\detokenize{ch/principles-second:physics-hs-thermodynamics-principles-second-composite}}
\sphinxAtStartPar
\sphinxstylestrong{todo} definition of a composite system. Heat conduction occurs between subsystems.

\sphinxAtStartPar
Entropy in classical thermodynamics is an extensive physical quantity: the entropy of a system composed of \(N\) simple subsystems is the sum of the entropies of the subsystems,
\begin{equation*}
\begin{split}S = \sum_{n=1:N} S_n \ .\end{split}
\end{equation*}
\sphinxAtStartPar
The entropy balance of a single subsystem that exchanges heat with the other subsystems and the external environment is written as
\begin{equation*}
\begin{split}\begin{aligned}
    dS_i & = \dfrac{\delta Q^{ext,i}_i}{T_i} + \dfrac{\delta^+ D_i}{T_i} = \\
         & = \dfrac{\delta Q^{ext}_i}{T_i} + \dfrac{\sum_{k \ne i} \delta Q_{ik}}{T_i} + \dfrac{\delta^+ D_i}{T_i} \ge \\
         & \ge \dfrac{\delta Q^{ext}_i}{T_i} + \dfrac{\sum_{k \ne i} \delta Q_{ik}}{T_i} \ . 
  \end{aligned}\end{split}
\end{equation*}
\sphinxAtStartPar
The entropy balance of the entire system is obtained by summing the entropy balances of the individual subsystems,
\begin{equation*}
\begin{split}\begin{aligned}
    dS & = \sum_i d S_i \ge \\
       & \ge \sum_i \left\{ \dfrac{\delta Q^{ext}_i}{T_i} + \dfrac{\sum_{k \ne i} \delta Q_{ik}}{T_i} \right\} = \\
       & = \sum_i \dfrac{\delta Q^{ext}_i}{T_i} + \underbrace{\sum_{\left\{i,k\right\}} \delta Q_{ik} \left( \dfrac{1}{T_i} - \dfrac{1}{T_k} \right)}_{\ge 0} \ge \\
       & \ge \sum_i \dfrac{\delta Q^{ext}_i}{T_i} \ . 
  \end{aligned}\end{split}
\end{equation*}
\sphinxAtStartPar
Using the relation that represents the natural tendency of heat transfer “from a system at a higher temperature to a system at a lower temperature,”
\begin{equation*}
\begin{split}\delta Q_{ik} \left( \dfrac{1}{T_i} - \dfrac{1}{T_k} \right) \ge 0 \ .\end{split}
\end{equation*}
\sphinxAtStartPar
\sphinxstylestrong{todo} \sphinxstyleemphasis{add reference to the natural tendency in heat transfer}


\subsection{Increase of Entropy in the Universe}
\label{\detokenize{ch/principles-second:increase-of-entropy-in-the-universe}}\label{\detokenize{ch/principles-second:physics-hs-thermodynamics-principles-second-universe}}
\sphinxAtStartPar
If we consider the universe as the closed and isolated system (but is this true? Who knows? Maybe it’s reasonable for now, but many things that seem reasonable today might be nonsense in a few years) consisting of a system of interest \(sys\) and the external environment \(env\).

\sphinxAtStartPar
The variation of entropy in the universe is the sum of the variation in the system and in the external environment. We denote by \(\delta Q_{sys,env}\) the heat flux which, if positive, increases the energy of the system and decreases that of the external environment. Assuming that the two subsystems are internally homogeneous,
\begin{equation*}
\begin{split}\begin{aligned}
d S^{univ} & = d S^{sys} + d S^{env} = \\
           & = \dfrac{\delta Q_{sys,env}}{T^{sys}} + \dfrac{\delta Q_{env,sys}}{T^{env}} = \\
           & = \dfrac{\delta Q_{sys,env}}{T^{sys}} - \dfrac{\delta Q_{sys,env}}{T^{env}} = \\
           & = \delta Q_{sys,env} \left( \dfrac{1}{T^{sys}} - \dfrac{1}{T^{env}} \right) \ge 0 \ ,
\end{aligned}\end{split}
\end{equation*}
\sphinxAtStartPar
we obtain the relation
\begin{equation*}
\begin{split}dS^{univ} \ge 0 \ ,\end{split}
\end{equation*}
\sphinxAtStartPar
which predicts the “non\sphinxhyphen{}decrease” of the entropy of the universe.


\subsection{Composite Continuous Systems}
\label{\detokenize{ch/principles-second:composite-continuous-systems}}\label{\detokenize{ch/principles-second:physics-hs-thermodynamics-principles-second-continuum}}
\sphinxstepscope


\section{Open Systems}
\label{\detokenize{ch/principles-open:open-systems}}\label{\detokenize{ch/principles-open:physics-hs-thermodynamics-foundation-principles-open}}\label{\detokenize{ch/principles-open::doc}}
\sphinxAtStartPar
In general, the balance equation for a physical quantity in an open system is derived from the balance of the same physical quantity for a closed system, adding the contribution of the flux terms of the desired physical quantity through the system’s boundary. Thus, if the balance of the physical quantity \(F\) for the closed system within volume \(V_t\) can be written as
\begin{equation*}
\begin{split}\dfrac{d}{dt} F_{V_t} = R^e_{V_t} \ ,\end{split}
\end{equation*}
\sphinxAtStartPar
the balance of the same physical quantity for an open system identified by the (geometric) volume \(v_t\) can be written as
\begin{equation*}
\begin{split}\frac{d}{dt} F_{v_t} = R^e_{v_t} - \Phi_{\partial v_t}(f) \ , \end{split}
\end{equation*}
\sphinxAtStartPar
where \(f\) is defined as the specific quantity of \(F\) per unit mass. The flux term through the boundary \(\partial v_t\) can be written as the sum of the flux contributions through portions \(s_{k,t}\) of the surface \(\partial v_t = \cup_k s_{k,t}\),
\begin{equation*}
\begin{split}\Phi_{\partial v_t}(f) = \sum_{s_{k,t}} \dot{m}_k \, f_k \ ,\end{split}
\end{equation*}
\sphinxAtStartPar
where \(\dot{m}_k = \rho_k vi^{rel}_{n,k}\) is the mass flux through the surface \(s_{k,t}\), and it is assumed that the quantity \(f_k\) is constant over the surface \(s_{k,t}\), or that the average value over the surface has been considered.

\begin{sphinxadmonition}{note}{Note:}
\sphinxAtStartPar
In case the quantity \(f\) is not uniform over the domain boundary and varies continuously, the flux term can be written as the summation of infinite terms through surfaces whose area tends to zero, via a surface integral, following the definition of a Riemann integral.
\end{sphinxadmonition}

\begin{sphinxadmonition}{note}{Note:}
\sphinxAtStartPar
A balance equation of a physical quantity for an open system also includes the balance equation of the same physical quantity for a closed system as a special case where the mass flux through the domain boundary is zero, \(\dot{m}_k = 0\).
\end{sphinxadmonition}

\sphinxAtStartPar
Here, we consider the balances (integrals, global of a system) of some fundamental physical quantities in classical mechanics: mass, momentum, angular momentum, and total energy.


\subsection{Mass Balance}
\label{\detokenize{ch/principles-open:mass-balance}}
\sphinxAtStartPar
The mass balance, \(F = M\), \(f = 1\), for an open system is
\begin{equation*}
\begin{split}\frac{d}{dt} M_{v_t} = - \sum_k \dot{m}_k\end{split}
\end{equation*}

\subsection{Momentum Balance}
\label{\detokenize{ch/principles-open:momentum-balance}}
\sphinxAtStartPar
 \sphinxstylestrong{todo} \sphinxstyleemphasis{reference to or from classical mechanics}


\subsection{Angular Momentum Balance}
\label{\detokenize{ch/principles-open:angular-momentum-balance}}
\sphinxAtStartPar
 \sphinxstylestrong{todo} \sphinxstyleemphasis{reference to or from classical mechanics}


\subsection{Total Energy Balance}
\label{\detokenize{ch/principles-open:total-energy-balance}}
\sphinxAtStartPar
The total energy balance \(F = E^{tot} = E + K\), \(f = e^{tot} = e + \frac{|\vec{v}|^2}{2}\),
\begin{equation*}
\begin{split}\frac{d}{dt} E^{tot}_{v_t} = P^{ext}_{v_t}  + \dot{Q}^{ext}_{v_t}  - \sum_k \dot{m}_k \, e^{tot}_k \ .\end{split}
\end{equation*}
\sphinxAtStartPar
The power of external actions \(P^{ext}_{v_t}\) can be written as the sum of contributions on the surfaces of the system’s boundary through which there is mass flow, and impermeable surfaces that can be used to extract work from the system. If the effects of viscous stresses on the surfaces through which there is mass flow can be neglected, the power of the actions on the system can be written as
\begin{equation*}
\begin{split}\begin{aligned}
  P^{ext} & = P^{ext,mech} + P^{ext,\Phi} = \\
          & = P^{ext,mech} - \sum_{k} \dot{m}_k \frac{P_k}{\rho_k} \ , 
\end{aligned}\end{split}
\end{equation*}
\sphinxAtStartPar
and the total energy balance of the system becomes
\begin{equation*}
\begin{split}\dfrac{d}{dt} E^{tot}_{v_t} = P^{ext,mech}_{v_t} + \dot{Q}^{ext} - \sum_k \dot{m}_k h^{tot}_k \ ,\end{split}
\end{equation*}
\sphinxAtStartPar
where the specific total enthalpy is defined as \(h^{tot} = e^{tot} + \frac{P}{\rho} = e + \frac{P}{\rho} + \frac{|\vec{v}|^2}{2}\).
\label{ch/principles-open:thermodynamics:principles:open:ex:turbine}
\begin{sphinxadmonition}{note}{Example 16 (Turbine)}


\end{sphinxadmonition}
\label{ch/principles-open:thermodynamics:principles:open:ex:compressor}
\begin{sphinxadmonition}{note}{Example 17 (Compressor)}


\end{sphinxadmonition}
\label{ch/principles-open:thermodynamics:principles:open:ex:comb-chamber}
\begin{sphinxadmonition}{note}{Example 18 (Combustion Chamber)}


\end{sphinxadmonition}

\sphinxstepscope


\chapter{Thermodynamics potentials}
\label{\detokenize{ch/potentials:thermodynamics-potentials}}\label{\detokenize{ch/potentials:classical-thermodynamics-potentials}}\label{\detokenize{ch/potentials::doc}}
\sphinxAtStartPar
In this section, principles and the mathematical formalism of classical thermodynamics are reviewed.


\section{First principle}
\label{\detokenize{ch/potentials:first-principle}}
\sphinxAtStartPar
For an extensive system
\begin{equation*}
\begin{split}d E = T \, d S + \mathbf{F} \cdot d \mathbf{X} \ ,\end{split}
\end{equation*}
\sphinxAtStartPar
being \(E\) the internal energy, \(T\) temperature, \(S\) entropy, \(\mathbf{F}\) generalized force, \(\mathbf{X}\) generalized displacement.

\sphinxAtStartPar
Following Gibbs’ formulation, internal energy \(E\) can be written as a function of a limited set of independent state variables,
\begin{equation*}
\begin{split}E(S, \mathbf{X}) \ .\end{split}
\end{equation*}
\sphinxAtStartPar
Internal energy \(E\), entropy \(S\) and the generalized displacement \(\mathbf{X}\) are \sphinxstylestrong{extensive} physical quantities, and thus (\sphinxstylestrong{thus?}) the derivatives
\begin{equation*}
\begin{split}T := \left.\dfrac{\partial S}{\partial E}\right|_{\mathbf{X}} \quad , \quad \mathbf{F} := \left.\dfrac{\partial E}{\partial \mathbf{X}}\right|_S\end{split}
\end{equation*}
\sphinxAtStartPar
are \sphinxstylestrong{intensive} quantities. A discussion about the difference between the concept of \sphinxstyleemphasis{additivity} and \sphinxstyleemphasis{extensivity}%
\begin{footnote}[1]\sphinxAtStartFootnote
\sphinxhref{https://arxiv.org/pdf/cond-mat/0201134}{H.Touchette, When is a quantity additive, and when is it extensive?}
%
\end{footnote}.
Beside being extensive, internal energy in classical thermodynamics is an homogeneous function%
\begin{footnote}[2]\sphinxAtStartFootnote
\sphinxurl{https://physics.stackexchange.com/q/677855}
%
\end{footnote} of order 1 of its arguments, namely
\begin{equation*}
\begin{split}E(a \, S, a \, \mathbf{X}) = a \, E(S, \mathbf{X}) \ .\end{split}
\end{equation*}
\sphinxAtStartPar
\sphinxstylestrong{Euler’s theorem for homogeneous functions} holds.
\label{ch/potentials:theorem-0}
\begin{sphinxadmonition}{note}{Theorem 1 (Euler’s theorem for homogeneous functions)}



\sphinxAtStartPar
Let \(f(x_i)\) and homogeneous function of order \(m\), i.e.
\begin{equation}\label{equation:ch/potentials:eq:homogeneous-fun:def}
\begin{split}f(a \, x_i) = a^m \, f(x_i) \ .\end{split}
\end{equation}
\sphinxAtStartPar
It follows that
\begin{equation*}
\begin{split}f(x_k) = m \, x_i \, \dfrac{\partial f}{\partial x_i}(x_k) \ .\end{split}
\end{equation*}\end{sphinxadmonition}

\sphinxAtStartPar
Proof immediately follows, evaluating the derivative of \eqref{equation:ch/potentials:eq:homogeneous-fun:def} w.r.t. \(a\), and evaluating for \(a = 1\), i.e.
\begin{equation*}
\begin{split}x_i \, \dfrac{\partial f}{\partial x_i} (a \, x_k) = m \, a^{m-1} \, f(x_k) \ ,\end{split}
\end{equation*}
\sphinxAtStartPar
and for \(a = 1\),
\begin{equation*}
\begin{split}f(x_k) = x_i \, \dfrac{\partial f}{\partial x_i} (x_k)\ .\end{split}
\end{equation*}
\sphinxAtStartPar
Thus, internal energy can be written as
\begin{equation*}
\begin{split}E(S, \mathbf{X}) = T \, S + \mathbf{F} \cdot \mathbf{X} \ .\end{split}
\end{equation*}

\subsection{First principle for different systems}
\label{\detokenize{ch/potentials:first-principle-for-different-systems}}\begin{itemize}
\item {} 
\sphinxAtStartPar
Single\sphinxhyphen{}component fluid, \(E(S, V, N)\)
\begin{equation*}
\begin{split}d E = T \, d S - P \, d V + \mu \, d N\end{split}
\end{equation*}
\item {} 
\sphinxAtStartPar
Multi\sphinxhyphen{}component fluid, \(E(S, V, N_k)\)
\begin{equation*}
\begin{split}d E = T \, dS - P \, d V + \mu_k \, d N_k\end{split}
\end{equation*}
\sphinxAtStartPar
where the change of number of particles (or moles, it depends on the description \sphinxhyphen{} anyway a non\sphinxhyphen{}dimensional number) \(d N_k\) is governed by the stoichiometric ratios of the reactions occurring in the system.

\item {} 
\sphinxAtStartPar
Single\sphinxhyphen{}component solid, \(E(S, \mathbf{X})\)
\begin{equation*}
\begin{split}d E = T \, dS + \mathbf{F} \cdot d \mathbf{X}\end{split}
\end{equation*}
\end{itemize}


\subsection{First principle for specific quantities}
\label{\detokenize{ch/potentials:first-principle-for-specific-quantities}}
\sphinxAtStartPar
First principle and thermodynamics can be writtenin terms of specific quantites, usually either for unit volume or for unit mass.


\subsubsection{First principle per unit mass}
\label{\detokenize{ch/potentials:first-principle-per-unit-mass}}
\sphinxAtStartPar
All the extensive quantites are written as the product of the mass of the system \(M\) and its density, namely
\begin{equation*}
\begin{split}E = M  e \quad , \quad S = M  s \quad , \quad \mathbf{X} = M \mathbf{x} \ ,\end{split}
\end{equation*}
\sphinxAtStartPar
and, using the product rule for differential \(d E = d (M e) = dM \, e + M \, de\), first principle can be written as
\begin{equation*}
\begin{split}0 = d M \underbrace{\left( e - T s - \mathbf{F} \cdot \mathbf{x} \right)}_{= 0} + M \left( d e - T ds - \mathbf{F} \cdot d \mathbf{x} \right) \ .\end{split}
\end{equation*}
\sphinxAtStartPar
The first term is identically zero, since it’s the expression of the internal energy divided by \(M\). Being \(M > 0\), the second term gives the first principle per unit mass
\begin{equation*}
\begin{split}d e = T ds + \mathbf{F} \cdot d \mathbf{x} \ .\end{split}
\end{equation*}

\paragraph{Different systems}
\label{\detokenize{ch/potentials:different-systems}}\begin{itemize}
\item {} 
\sphinxAtStartPar
Single\sphinxhyphen{}component fluid;
\begin{equation*}
\begin{split}V = M \dfrac{1}{\rho} \quad , \quad N = M \dfrac{1}{m} \ , \end{split}
\end{equation*}
\sphinxAtStartPar
being \(\rho\) the mass density and \(m\) the mass of a particle (or mole, it depends on the description) of the medium; \(m\) is \sphinxstylestrong{constant}. First principle becomes
\begin{equation*}
\begin{split}0 = d M \underbrace{\left( e - T s + \dfrac{P}{\rho} - \dfrac{\mu}{m} \right)}_{= 0} + M \left( d e - T ds - \dfrac{P}{\rho^2} d\rho \right) \ ,\end{split}
\end{equation*}
\sphinxAtStartPar
and thus
\begin{equation*}
\begin{split}de = T ds + \dfrac{P}{\rho^2} d \rho \ .\end{split}
\end{equation*}
\item {} 
\sphinxAtStartPar
Multi\sphinxhyphen{}component fluid
\begin{equation*}
\begin{split}V = M \dfrac{1}{\rho} \quad , \quad N_k = M_k \dfrac{1}{m}_k = M \dfrac{M_k}{M} \dfrac{1}{m_k} = M \dfrac{1}{m_k} \dfrac{\rho_k}{\rho} = M \dfrac{1}{m_k} w_k \ , \end{split}
\end{equation*}
\sphinxAtStartPar
being \(\rho\) the mass density and \(m\) the mass of a particle (or mole, it depends on the description) of the \(k^{th}\) substance; \(m_k\) is \sphinxstylestrong{constant}. The first principle becomes
\begin{equation*}
\begin{split}0 = d M \underbrace{\left( e - T s + \dfrac{P}{\rho} - \dfrac{\mu_k}{m_k} w_k \dfrac{}{} \right)}_{= 0} + M \left[ d e - T ds - \dfrac{P}{\rho^2} d\rho - \dfrac{\mu_k}{m_k} d w_k \right] \ ,\end{split}
\end{equation*}
\sphinxAtStartPar
and thus
\begin{equation*}
\begin{split}de = T ds + \dfrac{P}{\rho^2} d \rho + \dfrac{\mu_k}{m_k} d w_k\ .\end{split}
\end{equation*}
\item {} 
\sphinxAtStartPar
Single\sphinxhyphen{}component solid
\begin{equation*}
\begin{split}\text{\textbf{todo}}\end{split}
\end{equation*}
\end{itemize}


\subsubsection{First principle per unit volume}
\label{\detokenize{ch/potentials:first-principle-per-unit-volume}}
\sphinxAtStartPar
All the extensive quantites are written as the product of the mass of the system \(M\) and its density, namely
\begin{equation*}
\begin{split}E = V  \mathcal{E} \quad , \quad S = M  \mathcal{S} \quad , \quad \mathbf{X} = M \symbf{\chi} \ ,\end{split}
\end{equation*}
\sphinxAtStartPar
and, using the product rule for differential \(d E = d (V \mathcal{E}) = dV \, \mathcal{E} + V \, d\mathcal{E}\), first principle can be written as
\begin{equation*}
\begin{split}0 = d V \underbrace{\left( \mathcal{E} - T \mathcal{S} - \mathbf{F} \cdot \symbf{\chi} \right)}_{= 0} + V \left( d \mathcal{E} - T d \mathcal{S} - \mathbf{F} \cdot d \symbf{\chi} \right) \ .\end{split}
\end{equation*}
\sphinxAtStartPar
The first term is identically zero, since it’s the expression of the internal energy divided by \(V\). Being \(V > 0\), the second term gives the first principle per unit volume
\begin{equation*}
\begin{split}d \mathcal{E} = T d\mathcal{S} + \mathbf{F} \cdot d \symbf{\chi} \ .\end{split}
\end{equation*}

\paragraph{Different systems}
\label{\detokenize{ch/potentials:id3}}\begin{itemize}
\item {} 
\sphinxAtStartPar
Single\sphinxhyphen{}component fluid;
\begin{equation*}
\begin{split}V = V \cdot 1 \quad , \quad N = V \dfrac{M}{V} \dfrac{1}{m} = V \dfrac{\rho}{m} \ , \end{split}
\end{equation*}
\sphinxAtStartPar
being \(\rho\) the mass density and \(m\) the mass of a particle (or mole, it depends on the description) of the medium; \(m\) is \sphinxstylestrong{constant}. First principle becomes
\begin{equation*}
\begin{split}0 = d V \underbrace{\left( \mathcal{E} - T \mathcal{S} + P - \mu \dfrac{\rho}{m} \right)}_{= 0} + V \left( d \mathcal{E} - T d\mathcal{S} + 0 - \frac{\mu}{m} \, d \rho  \right) \ ,\end{split}
\end{equation*}
\sphinxAtStartPar
and thus
\begin{equation*}
\begin{split}d \mathcal{E} = T d \mathcal{S} + \dfrac{\mu}{m} d \rho \ .\end{split}
\end{equation*}
\sphinxAtStartPar
This latter formulation is consistent with the principle per unit mass. Volume density can be written as the product of mass density and the mass density of the physical quantity of interest, namely \(\mathcal{E} = \rho e\)
\begin{equation*}
\begin{split}\begin{aligned}
    0 & = - d( \rho e ) + T d (\rho s) + \frac{\mu}{m} d \rho = \\
      & = d \rho \left( - e + T s + \frac{\mu}{m} \right) - \rho \left( d e + T d s \right) = \\
      & = d \rho \underbrace{\left( - e - \dfrac{P}{\rho} + T s + \frac{\mu}{m} \right)}_{\text{def. of $e$}} + \rho \underbrace{\left( - d e + \dfrac{P}{\rho^2} d \rho + T d s \right)}_{1^{st} \text{ pr. per unit mass}} \ .
  \end{aligned}\end{split}
\end{equation*}
\item {} 
\sphinxAtStartPar
Multi\sphinxhyphen{}component fluid
\begin{equation*}
\begin{split}\text{\textbf{todo}}\end{split}
\end{equation*}
\item {} 
\sphinxAtStartPar
Single\sphinxhyphen{}component solid
\begin{equation*}
\begin{split}\text{\textbf{todo}}\end{split}
\end{equation*}
\end{itemize}


\section{Potentials \sphinxhyphen{} specific quantities}
\label{\detokenize{ch/potentials:potentials-specific-quantities}}
\sphinxAtStartPar
Here mechanical \(\mathbf{X}_m\) and non\sphinxhyphen{}mechanical \(\mathbf{X}_n\) generalized forces and displacements are recognized to define enthaply and Gibbs’ free energy later.
\begin{equation*}
\begin{split}e(s, \mathbf{x}) = e \left( \frac{S}{M}, \frac{\mathbf{X}}{M} \right) = \frac{1}{M} E\left( \frac{S}{M}, \frac{\mathbf{X}}{M} \right) \qquad \text{\textbf{todo check}} \end{split}
\end{equation*}
\sphinxAtStartPar
\sphinxstylestrong{Internal energy, \(e(s, \mathbf{x})\).}
\begin{equation*}
\begin{split}d e = T ds + \mathbf{F} \cdot d \mathbf{x}\end{split}
\end{equation*}
\sphinxAtStartPar
\sphinxstylestrong{Helmholtz free energy, \(f(T, \mathbf{x}) = e - T s\).}
\begin{equation*}
\begin{split}\begin{aligned}
  d f 
  & = de - T d s - s d T= \\
  & =- s d T + \mathbf{F} \cdot d \mathbf{x} \ .
\end{aligned}\end{split}
\end{equation*}
\sphinxAtStartPar
\sphinxstylestrong{Enthaply, \(h(T, \mathbf{F}_m, \mathbf{x}_n) = e - \mathbf{F}_m \cdot \mathbf{x}_m\).}
\begin{equation*}
\begin{split}\begin{aligned}
  d h 
  & = de - d \mathbf{F}_m \cdot \mathbf{x}_m - \mathbf{F}_m \cdot d \mathbf{x}_m = \\
  & = T d s - \mathbf{x}_m \cdot d \mathbf{F}_m + \mathbf{F}_n \cdot d \mathbf{x}_n \ .
\end{aligned}\end{split}
\end{equation*}
\sphinxAtStartPar
\sphinxstylestrong{Gibbs’ free energy, \(g(T, \mathbf{F}_m, \mathbf{x}_n) = h - T s = e - \mathbf{F}_m \cdot \mathbf{x}_m - T s = f - \mathbf{F}_m \cdot \mathbf{x}_m = \mathbf{F}_n \cdot \mathbf{x}_n\).}
\begin{equation*}
\begin{split}\begin{aligned}
  d g 
  & = dh - T ds - s dT = \\
  & = - s  dT - \mathbf{x}_m \cdot d \mathbf{F}_m + \mathbf{F}_n \cdot d \mathbf{x}_n = \\
  & = \mathbf{x}_n \cdot d \mathbf{F}_n + \mathbf{F}_n \cdot d \mathbf{x}_n \ .
\end{aligned}\end{split}
\end{equation*}
\sphinxAtStartPar
\sphinxstylestrong{Partial derivatives of potentials.}
\begin{equation*}
\begin{split}\begin{aligned}
 T & = \quad \left.\frac{\partial e}{\partial s}\right|_{\mathbf{x}} && = \quad \left.\frac{\partial h}{\partial s}\right|_{\mathbf{F}_m,\mathbf{x}_n} \\
 s & =     - \left.\frac{\partial f}{\partial T}\right|_{\mathbf{x}} && =     - \left.\frac{\partial g}{\partial T}\right|_{\mathbf{F}_m,\mathbf{x}_n} \\
 \mathbf{F}_m & = \quad \left.\frac{\partial e}{\partial \mathbf{x}_m}\right|_{s,\mathbf{x}_n} && = \quad \left.\frac{\partial f}{\partial \mathbf{x}_m}\right|_{T,\mathbf{x}_n} \\
 \mathbf{F}_n & = \quad  \left.\frac{\partial e}{\partial \mathbf{x}_n}\right|_{s,\mathbf{x}_m} && = \quad \left.\frac{\partial f}{\partial \mathbf{x}_n}\right|_{T,\mathbf{x}_m} && = \quad  \left.\frac{\partial h}{\partial \mathbf{x}_n}\right|_{s,\mathbf{F}_m} && = \quad \left.\frac{\partial g}{\partial \mathbf{x}_n}\right|_{T,\mathbf{F}_m} \\
 \mathbf{x}_m & = -\left.\frac{\partial h}{\partial \mathbf{F}_m}\right|_{s,\mathbf{x}_n} && =- \left.\frac{\partial g}{\partial \mathbf{F}_m}\right|_{T,\mathbf{x}_n}\\
\end{aligned}\end{split}
\end{equation*}
\sphinxAtStartPar
\sphinxstylestrong{Maxwell’s relations.}
\begin{equation*}
\begin{split}\text{\textbf{todo} Uncomment}\end{split}
\end{equation*}



\bigskip\hrule\bigskip


\sphinxstepscope




\chapter{Thermodynamic coefficients}
\label{\detokenize{ch/coefficients:thermodynamic-coefficients}}\label{\detokenize{ch/coefficients:classical-thermodynamics-coefficients}}\label{\detokenize{ch/coefficients::doc}}
\sphinxAtStartPar
In this section different related to first derivatives of thermodynamic state variables are introduced and discussed for different systems.


\section{Thermodynamic coefficients of a single\sphinxhyphen{}component fluid}
\label{\detokenize{ch/coefficients:thermodynamic-coefficients-of-a-single-component-fluid}}
\sphinxAtStartPar
For a 1\sphinxhyphen{}component fluid with constant composition, the first principle reads
\begin{equation*}
\begin{split}de = T ds - P dv = T ds + \frac{P}{\rho^2} d \rho \ .\end{split}
\end{equation*}
\sphinxAtStartPar
\sphinxstylestrong{Heat capacity.}
\begin{equation*}
\begin{split}c_x := T \dfrac{\partial s}{\partial T}\Big|_x\end{split}
\end{equation*}
\sphinxAtStartPar
For fluid systems, usually heat capacity at constant pressure or constant density are the most used.

\sphinxAtStartPar
\sphinxstylestrong{Thermal expansion coefficients.}
\begin{equation*}
\begin{split}\alpha_x := \dfrac{1}{v} \dfrac{\partial v}{\partial T}\Big|_x = - \dfrac{1}{\rho} \dfrac{\partial \rho}{\partial T}\Big|_x\end{split}
\end{equation*}
\sphinxAtStartPar
For fluid systems, usually thermal expansion coefficent at constant pressure is the most used.

\sphinxAtStartPar
\sphinxstylestrong{Compressibility coefficients.}
\begin{equation*}
\begin{split}\beta_x := - \dfrac{1}{v} \dfrac{\partial v}{\partial P}\Big|_x = \dfrac{1}{\rho} \dfrac{\partial \rho}{\partial P}\Big|_x\end{split}
\end{equation*}
\sphinxAtStartPar
For fluid systems, usually compressibility coefficent at constant temperature or entropy are the most used.


\section{Relations between thermodynamic coefficients}
\label{\detokenize{ch/coefficients:relations-between-thermodynamic-coefficients}}
\sphinxAtStartPar
\sphinxstylestrong{Relation between \(c_v\) and \(c_p\).}
\begin{equation}\label{equation:ch/coefficients:eq:cv-cp}
\begin{split}c_P - c_v = T \, v \, \dfrac{\alpha_P^2}{\beta_T} \ . \end{split}
\end{equation}
\sphinxAtStartPar
\sphinxstylestrong{Relation between \(\beta_s\) and \(\beta_T\) \sphinxhyphen{} 1}
\begin{equation}\label{equation:ch/coefficients:eq:beta-1}
\begin{split}\beta_s = \frac{c_v}{c_P} \beta_T \ .\end{split}
\end{equation}
\sphinxAtStartPar
\sphinxstylestrong{Relation between \(\beta_s\) and \(\beta_T\) \sphinxhyphen{} 2}
\begin{equation}\label{equation:ch/coefficients:eq:beta-2}
\begin{split}\beta_s - \beta_T = - \frac{v T}{c_P} \alpha_P^2 \ .\end{split}
\end{equation}
\sphinxAtStartPar
\sphinxstylestrong{Relation between \(\beta_s\) and \(\beta_T\) \sphinxhyphen{} 3}
\begin{equation}\label{equation:ch/coefficients:eq:beta-3}
\begin{split}\dfrac{1}{\beta_s} - \dfrac{1}{\beta_T} = \dfrac{T}{v c_v} \left( \partial_T P|_v \right)^2 \ .\end{split}
\end{equation}\subsubsection*{Proof of the relation between heat capacities \protect\(c_v\protect\), \protect\(c_P\protect\)}

\sphinxAtStartPar
Changing independent variables from \((T,v)\) to \((P,v)\) in the expression of entropy \(s(T,v) = s(T, P(v,T))\),
\begin{equation*}
\begin{split}\begin{aligned}
  \dfrac{c_v}{T} := \left.\dfrac{\partial s}{\partial T}\right|_v 
  & = \left.\dfrac{\partial s}{\partial T}\right|_P + \left.\dfrac{\partial s}{\partial P}\right|_T \left.\dfrac{\partial P}{\partial T}\right|_v = & \qquad \text{(Maxwell $\partial_P s|_T = - \partial_T v|_P$)} \\
  & = \left.\dfrac{\partial s}{\partial T}\right|_P - \left.\dfrac{\partial v}{\partial T}\right|_P \left.\dfrac{\partial P}{\partial T}\right|_v = & \qquad \text{(relation below $\partial_T P|_v = \dots$)} \\
  & = \partial_T s|_P + \dfrac{\left( \partial_P v |_T \right)^2}{\partial_P v |_T} = & \qquad \text{(def of TD coeffs)} \\
  & = \dfrac{c_P}{T} - \dfrac{ v \alpha_P^2}{\beta_T} 
\end{aligned}\end{split}
\end{equation*}
\sphinxAtStartPar
having exploited the relation
\begin{equation*}
\begin{split}\begin{aligned}
  \left.\dfrac{\partial P}{\partial T}\right|_v 
    = \dfrac{\partial(P,v)}{\partial (T,v)} 
    = \dfrac{\partial(P,v) / \partial(T,P)}{\partial (T,v) / \partial(T,P)} 
    = - \dfrac{\partial_P v|_T}{\partial_P v|_T} \ .
\end{aligned}\end{split}
\end{equation*}\subsubsection*{Proof of the relation between \protect\(\beta_s\protect\) and \protect\(\beta_T\protect\) \sphinxhyphen{} 1}
\begin{equation*}
\begin{split}\begin{aligned}
 v \beta_T & = \partial_P v |_T \\
 v \beta_s & = \partial_P v |_s 
 = \dfrac{\partial(v,s)}{\partial(P,s)} \dfrac{\partial(v,T)}{\partial(v,T)}  \dfrac{\partial(P,T)}{\partial(P,T)} 
 = \underbrace{\dfrac{\partial(P,T)}{\partial(P,s)}}_{= \dfrac{1}{\partial_T s|_P}} \underbrace{\dfrac{\partial(v,T)}{\partial(P,T)}}_{\partial_P v|_T} \underbrace{\dfrac{\partial(v,s)}{\partial(v,T)}}_{\partial_T s|_v} 
 = \dfrac{c_v}{c_P} \left( v \beta_T \right)
\end{aligned}\end{split}
\end{equation*}\subsubsection*{Proof of the relation between \protect\(\beta_s\protect\) and \protect\(\beta_T\protect\) \sphinxhyphen{} 2}
\begin{equation*}
\begin{split}
\beta_s = \frac{c_v}{c_P} \beta_T 
  = \dfrac{\beta_T}{c_P} \left[ c_P - T v \frac{\alpha^2_P}{\beta_T} \right]
  = \beta_T - T v \frac{\alpha^2_P}{c_P} \ .
\end{split}
\end{equation*}\subsubsection*{Proof of the relation between \protect\(\beta_s\protect\) and \protect\(\beta_T\protect\) \sphinxhyphen{} 3}
\begin{equation*}
\begin{split}\begin{aligned}
\dfrac{1}{\beta_s} = \frac{c_P}{c_v} \dfrac{1}{\beta_T}
  & = \dfrac{1}{c_v \, \beta_T} \left[ c_v + T v \frac{\alpha^2_P}{\beta_T} \right] = \\
  & = \dfrac{1}{\beta_T} \, \dfrac{1}{c_v} \left[ c_v + T v \frac{\alpha^2_P}{\beta_T} \right] = \\
  & = \dfrac{1}{\beta_T} + T v \frac{\frac{1}{v^2} (\partial_T v|_P)^2}{-\frac{1}{v} \partial_P v|_T} \frac{1}{-\frac{1}{v} \partial_P v|_T} = \\
  & = \dfrac{1}{\beta_T} + T v \left( \frac{\partial_T v|_P}{\partial_P v|_T} \right)^2
    = \dfrac{1}{\beta_T} + T v \left( \frac{\partial (v,P)}{\partial (T,P)} \frac{\partial (P,T)}{\partial (v,T)} \right)^2 = \dfrac{1}{\beta_T} + T v \left( \left.\dfrac{\partial P}{\partial T}\right|_v  \right)^2
\end{aligned}\end{split}
\end{equation*}

\section{Thermodynamic equilibrium}
\label{\detokenize{ch/coefficients:thermodynamic-equilibrium}}
\sphinxAtStartPar
Here thermodynamic equilibrium is discussed for a single\sphinxhyphen{}component fluid, for which the first principle reads
\begin{equation*}
\begin{split}d e = T ds - P dv = T ds + \frac{P}{\rho^2} d \rho \ .\end{split}
\end{equation*}
\sphinxAtStartPar
First conditions on energy, \(e(s, \rho)\), as a function of entropy and density then the equivalent conditions on entropy \(s(e, \rho)\), as a function of energy and density are discussed.


\subsection{Conditions on energy}
\label{\detokenize{ch/coefficients:conditions-on-energy}}
\sphinxAtStartPar
Thermodynamic equilibrium implies conditions on the second order term in series expansion of the function \(e(s, \rho)\) (\sphinxstylestrong{todo} \sphinxstyleemphasis{why? Spend few words. Use Landau as a reference if needed})
\begin{equation*}
\begin{split}\begin{aligned}
  e(s+ds, \rho+d\rho) 
  & = e(s, \rho) + \partial_s e|_{\rho} (s,\rho) d \rho + \partial_\rho e|_{s} (s, \rho) ds + \\
  & + \dfrac{1}{2} \left[ \partial_{ss} e|_{\rho}(s,\rho) ds^2 + 2 \partial_{s\rho} e(s, \rho) ds d\rho + \partial_{\rho \rho} e(s, \rho) d \rho^2 \right] \ .
\end{aligned}\end{split}
\end{equation*}
\sphinxAtStartPar
The equilibrium energy must be a \sphinxstyleemphasis{minimum} (\sphinxstylestrong{todo} check the function that must be a \sphinxstyleemphasis{minimum}, since \(\partial_s e|_{\rho} = T\),… and \(T\) not zero, and thus it can be a minimum!): the second order term must be positive for any increment of independent physical variables \(d \rho\), \(d s\), i.e.
\begin{equation*}
\begin{split}\dfrac{1}{2} \left[ \partial_{ss} e|_{\rho} ds^2 + 2 \partial_{s\rho} e ds d\rho + \partial_{\rho \rho} e|_{s} d \rho^2 \right] \end{split}
\end{equation*}
\sphinxAtStartPar
is a positive\sphinxhyphen{}definite quadratic form, i.e. the Hessian
\begin{equation*}
\begin{split}\begin{bmatrix} \partial_{ss} e|_{\rho} & \partial_{s \rho} e \\ \partial_{s \rho } e & \partial_{\rho \rho} e|_{s} \end{bmatrix} \end{split}
\end{equation*}
\sphinxAtStartPar
is definite positive, and thus teh following conditions must hold
\begin{equation*}
\begin{split}\begin{cases}
  \partial_{ss} e|_{\rho} > 0 \\
  \partial_{ss} e|_{\rho} \partial_{\rho \rho} e|_{s} - \left( \partial_{s \rho} e \right)^2 < 0
\end{cases}\end{split}
\end{equation*}

\subsection{Conditions on entropy}
\label{\detokenize{ch/coefficients:conditions-on-entropy}}
\sphinxAtStartPar
Using \(e\), \(\rho\) as independent thermodynamic state variables, and the entropy \(s(e, \rho)\) it can be proved that the condition on \(e(s, \rho)\) for the thermodynamic equilibrium implies that \(s(e, \rho)\) is a \sphinxstyleemphasis{maxiumum} (\sphinxstylestrong{todo} \sphinxstyleemphasis{check the meaning of maximum here}), and thus
\begin{equation*}
\begin{split}\dfrac{1}{2} \left[ \partial_{ee} s|_{\rho} de^2 + 2 \partial_{e\rho} s de d\rho + \partial_{\rho \rho} s|_{e} d \rho^2 \right] \end{split}
\end{equation*}
\sphinxAtStartPar
is a positive\sphinxhyphen{}definite quadratic form, i.e. the Hessian
\begin{equation*}
\begin{split}\begin{bmatrix} \partial_{ee} s|_{\rho} & \partial_{e \rho} s \\ \partial_{e \rho } s & \partial_{\rho \rho} s|_{e} \end{bmatrix} \end{split}
\end{equation*}
\sphinxAtStartPar
is definite positive, and thus teh following conditions must hold
\begin{equation*}
\begin{split}\begin{cases}
  \partial_{ee} s|_{\rho} < 0 \\
  \partial_{ee} s|_{\rho} \partial_{\rho \rho} s|_{e} - \left( \partial_{e \rho} s \right)^2 < 0
\end{cases}\end{split}
\end{equation*}\subsubsection*{Relation between partial derivatives of \protect\(e(s,\rho)\protect\) and \protect\(s(e,\rho)\protect\)}

\sphinxAtStartPar
The relation
\begin{equation*}
\begin{split}e = e_{s \rho}(s_{e \rho}(e, \rho), \rho) \ ,\end{split}
\end{equation*}
\sphinxAtStartPar
provides the link between the two representations, having written \(e_{s,\rho}()\), \(s_{e,\rho}()\) the functions with the arguent indicated as indices. This relation contains only \(e\), \(\rho\) as independent variables. All the required relations are evaluated computing partial derivatives of this relation.

\sphinxAtStartPar
\sphinxstylestrong{First\sphinxhyphen{}order derivatives.} It can be proved that
\begin{equation*}
\begin{split}\begin{aligned}
  \partial_e s |_{\rho} & = \dfrac{1}{\partial_s e |_{\rho}} \\
  \partial_{\rho} s |_e & = - \dfrac{\partial_{\rho} e |_{s}}{\partial_s e |_{\rho}} \\
\end{aligned}\end{split}
\end{equation*}
\sphinxAtStartPar
\sphinxstylestrong{Second\sphinxhyphen{}order derivatives.} It can be proved that
\begin{equation*}
\begin{split}\begin{aligned}
  \partial_{ee}       s|_{\rho} & = - \dfrac{1}{\left( \partial_s e|_{\rho} \right)^3} \partial_{ss} e |_{\rho} \\
  \partial_{e\rho}    s         & =   \dfrac{\partial_{\rho} e|_s}{\left( \partial_s e|_{\rho} \right)^3} \partial_{ss} e |_{\rho} - \dfrac{\partial_{s \rho} e}{(\partial_s e|_{\rho})^2} \\
  \partial_{\rho\rho} s|_{e}    & = - \dfrac{1}{\left( \partial_s e|_{\rho} \right)^3} \left[ \left( -\frac{\partial_\rho e_s}{\partial_s e|_\rho} \right)^2 \partial_{ss} e |_{\rho} + 2 \left(-\frac{\partial_\rho e_s}{\partial_s e|_\rho}\right) \partial_{s \rho} e + \partial_{\rho \rho} e|_s  \right] \\
\end{aligned}\end{split}
\end{equation*}\subsubsection*{Equivalence of conditions on \protect\(e(s, \rho)\protect\) and on \protect\(s(e, \rho)\protect\) for thermodynamic equilibrium}

\sphinxAtStartPar
Exploiting first condition on partial derivatives of \(e(s,\rho)\), the first condition on the partial derivatives of \(s(e,\rho)\) is
\begin{equation*}
\begin{split}\partial_{ee} s|_{\rho} = - \dfrac{1}{\left( \partial_s e|_{\rho} \right)^3} \partial_{ss} e |_{\rho} = - \dfrac{1}{T^3} \partial_{ss} e |_{\rho}< 0 \end{split}
\end{equation*}
\sphinxAtStartPar
since \(T > 0\) and \(\partial_{ss} e|_{\rho} > 0\).

\sphinxAtStartPar
Second condition for derivatives of \(s(e, \rho)\) is
\begin{equation*}
\begin{split}\begin{aligned}
  \partial_{\rho \rho} s|_{e} \partial_{ee} s |_{\rho} - \left( \partial_{\rho s} e \right)^2 
  & = \dots \\
  & = \frac{1}{T^4} \left[ \partial_{ss} e|_{\rho} \partial_{\rho \rho} e|_{s} - \left( \partial_{s \rho} e \right)^2  \right] < 0 \ .
\end{aligned} \end{split}
\end{equation*}


\sphinxstepscope




\chapter{Stati della materia e modelli}
\label{\detokenize{ch/media:stati-della-materia-e-modelli}}\label{\detokenize{ch/media:classical-thermodynamics-media}}\label{\detokenize{ch/media::doc}}
\sphinxAtStartPar
\sphinxstylestrong{Stati della materia.}
\begin{itemize}
\item {} 
\sphinxAtStartPar
gas

\item {} 
\sphinxAtStartPar
liquidi

\item {} 
\sphinxAtStartPar
solidi

\item {} 
\sphinxAtStartPar
plasma

\end{itemize}

\sphinxAtStartPar
\sphinxstylestrong{Alcune leggi costitutive.}
\begin{itemize}
\item {} 
\sphinxAtStartPar
solidi elastici:
\begin{itemize}
\item {} 
\sphinxAtStartPar
solidi lineari elastici isotropi

\end{itemize}

\item {} 
\sphinxAtStartPar
fluidi:
\begin{itemize}
\item {} 
\sphinxAtStartPar
in base all’equazione di stato:
\begin{itemize}
\item {} 
\sphinxAtStartPar
gas perfetti

\item {} 
\sphinxAtStartPar
gas reali

\item {} 
\sphinxAtStartPar
…

\end{itemize}

\item {} 
\sphinxAtStartPar
in base all’espressione degli sforzi:
\begin{itemize}
\item {} 
\sphinxAtStartPar
fluidi newtoniani

\item {} 
\sphinxAtStartPar
fluidi non\sphinxhyphen{}newtoniani

\end{itemize}

\end{itemize}

\end{itemize}

\sphinxstepscope

\begin{sphinxuseclass}{sd-container-fluid}
\begin{sphinxuseclass}{sd-sphinx-override}
\begin{sphinxuseclass}{sd-p-0}
\begin{sphinxuseclass}{sd-mt-2}
\begin{sphinxuseclass}{sd-mb-4}
\begin{sphinxuseclass}{sd-row}
\begin{sphinxuseclass}{sd-row-cols-2}
\begin{sphinxuseclass}{sd-gx-2}
\begin{sphinxuseclass}{sd-gy-1}
\begin{sphinxuseclass}{sd-col}
\begin{sphinxuseclass}{sd-d-flex-row}
\begin{sphinxuseclass}{sd-align-minor-center}
\begin{sphinxuseclass}{sd-container-fluid}
\begin{sphinxuseclass}{sd-sphinx-override}
\begin{sphinxuseclass}{sd-row}
\begin{sphinxuseclass}{sd-row-cols-2}
\begin{sphinxuseclass}{sd-row-cols-xs-2}
\begin{sphinxuseclass}{sd-row-cols-sm-3}
\begin{sphinxuseclass}{sd-row-cols-md-3}
\begin{sphinxuseclass}{sd-row-cols-lg-3}
\begin{sphinxuseclass}{sd-gx-3}
\begin{sphinxuseclass}{sd-gy-1}
\begin{sphinxuseclass}{sd-col}
\begin{sphinxuseclass}{sd-col-auto}
\begin{sphinxuseclass}{sd-d-flex-row}
\begin{sphinxuseclass}{sd-align-minor-center}
\sphinxAtStartPar
basics

\end{sphinxuseclass}
\end{sphinxuseclass}
\end{sphinxuseclass}
\end{sphinxuseclass}
\begin{sphinxuseclass}{sd-col}
\begin{sphinxuseclass}{sd-col-auto}
\begin{sphinxuseclass}{sd-d-flex-row}
\begin{sphinxuseclass}{sd-align-minor-center}
\sphinxAtStartPar
Jan 26, 2025

\end{sphinxuseclass}
\end{sphinxuseclass}
\end{sphinxuseclass}
\end{sphinxuseclass}
\begin{sphinxuseclass}{sd-col}
\begin{sphinxuseclass}{sd-col-auto}
\begin{sphinxuseclass}{sd-d-flex-row}
\begin{sphinxuseclass}{sd-align-minor-center}
\sphinxAtStartPar
0 min read

\end{sphinxuseclass}
\end{sphinxuseclass}
\end{sphinxuseclass}
\end{sphinxuseclass}
\end{sphinxuseclass}
\end{sphinxuseclass}
\end{sphinxuseclass}
\end{sphinxuseclass}
\end{sphinxuseclass}
\end{sphinxuseclass}
\end{sphinxuseclass}
\end{sphinxuseclass}
\end{sphinxuseclass}
\end{sphinxuseclass}
\end{sphinxuseclass}
\end{sphinxuseclass}
\end{sphinxuseclass}
\end{sphinxuseclass}
\end{sphinxuseclass}
\end{sphinxuseclass}
\end{sphinxuseclass}
\end{sphinxuseclass}
\end{sphinxuseclass}
\end{sphinxuseclass}
\end{sphinxuseclass}
\end{sphinxuseclass}

\section{Gas ideali}
\label{\detokenize{ch/ideal_gases:gas-ideali}}\label{\detokenize{ch/ideal_gases:classical-thermodynamics-ideal-gases}}\label{\detokenize{ch/ideal_gases::doc}}
\sphinxAtStartPar
\sphinxstylestrong{Legge di Boyle\sphinxhyphen{}Mariotte.} \(P V = \text{cost.}\) a \(T\) costante (trasformazione isoterma).

\sphinxAtStartPar
\sphinxstylestrong{Legge di Gay\sphinxhyphen{}Lussac I (o Charles).} \(V \propto T\) a \(P\) costante (trasformazione isobara).

\sphinxAtStartPar
\sphinxstylestrong{Legge di Gay\sphinxhyphen{}Lussac II.} \(P \propto T\) a \(V\) costante (trasformazione isocora).

\sphinxAtStartPar
\sphinxstylestrong{Legge dei gas ideali.} \(P V = n R T\)

\sphinxstepscope


\part{Thermal Engineering}

\sphinxstepscope


\chapter{Principles of Thermal Engineering}
\label{\detokenize{ch/thermal-engineering:principles-of-thermal-engineering}}\label{\detokenize{ch/thermal-engineering:thermal-engineering-intro}}\label{\detokenize{ch/thermal-engineering::doc}}
\sphinxAtStartPar
Thermal enginnering is a discipline dealing with applications and systems for energy conversion and transfer through work and heat, usually involving other branches or field in physics/engineering, like \DUrole{xref,myst}{thermodynamics}, \DUrole{xref,myst}{fluid mechanics}, \DUrole{xref,myst}{heat transfer}, \DUrole{xref,myst}{mass transfer}, \DUrole{xref,myst}{chemistry},…

\sphinxAtStartPar
Physical processes: continuum (usaully fluid) mechanics, mass and heat transfer, chemistry,…

\sphinxAtStartPar
Applications: power plants, heat engines (direct cycles); refrigerator systems (inverse cycles); heat exchangers; heating; heating, ventilation, and air conditioning (HVAC); thermal insulation,…
\subsubsection*{Thermodynamic transformations and cycles}
\subsubsection*{Heat transfer}

\sphinxstepscope




\chapter{Thermodynamic transformations and thermomechanical systems}
\label{\detokenize{ch/thermodynamic_transformations:thermodynamic-transformations-and-thermomechanical-systems}}\label{\detokenize{ch/thermodynamic_transformations:classical-thermodynamics-transformations}}\label{\detokenize{ch/thermodynamic_transformations::doc}}
\sphinxstepscope




\chapter{Thermodynamic cycles and heat engines}
\label{\detokenize{ch/heat_engines:thermodynamic-cycles-and-heat-engines}}\label{\detokenize{ch/heat_engines:classical-thermodynamics-heat-engines}}\label{\detokenize{ch/heat_engines::doc}}
\sphinxAtStartPar
\sphinxstylestrong{Direct and reversed thermodynamic cycles.} Direct cycles: transform heat to mechanical work. Reverset cycle: transform mechanical work to heat transfer (e.g. refrigeration/cooling \sphinxhyphen{} refrigerator cycles \sphinxhyphen{} or heating \sphinxhyphen{} heat pump cycles)

\sphinxAtStartPar
\sphinxstylestrong{Carnot cycle \sphinxhyphen{} reversible cycle between two constant temperatur sources.} Planck and Kelvin statements of \(2^{nd}\) principle of thermodynamics. Maximum efficiency….

\sphinxAtStartPar
\sphinxstylestrong{Ideal models of real cycles.} Otto, Diesel, Atkinson, Stirling (ICE), Rankine, Joule\sphinxhyphen{}Brayton

\sphinxAtStartPar
\sphinxstylestrong{Heat engines.} Reciprocating (piston)/reactive (turbine) engines



\sphinxstepscope




\chapter{Heat transfer}
\label{\detokenize{ch/heat-transfer:heat-transfer}}\label{\detokenize{ch/heat-transfer:classical-thermodynamics-heat-transmission}}\label{\detokenize{ch/heat-transfer::doc}}
\sphinxAtStartPar
Three main heat transfer mechanisms exist: conduction, convection, radiation.

\sphinxAtStartPar
\sphinxstylestrong{Conduction.} Main heat transfer mechanism in solid media: molecules of the solid are not free to move and conduction is a diffusion process of the microscopical thermal agitation of the molecules of the solid structure.

\sphinxAtStartPar
\sphinxstylestrong{Convection.} Main heat transfer machanism in fluid media: moldeculus of the fluid are free to move; as they move, they transport “thermal energy” and xechange heat with the surrounding environment.

\sphinxAtStartPar
\sphinxstylestrong{Radiation.} It’s the only heat transfer mechanism that needs no matter to occur, as it can occur in “vacuum”%
\begin{footnote}[1]\sphinxAtStartFootnote
Vacuum of matter, mass, but not physical properties, as the dielectric constant and magnetic permeability of free space/”vacuum”.
%
\end{footnote}


\bigskip\hrule\bigskip







\renewcommand{\indexname}{Proof Index}
\begin{sphinxtheindex}
\let\bigletter\sphinxstyleindexlettergroup
\bigletter{definition\sphinxhyphen{}0}
\item\relax\sphinxstyleindexentry{definition\sphinxhyphen{}0}\sphinxstyleindexextra{ch/principles\sphinxhyphen{}gibbs\sphinxhyphen{}phase\sphinxhyphen{}rule}\sphinxstyleindexpageref{ch/principles-gibbs-phase-rule:\detokenize{definition-0}}
\indexspace
\bigletter{definition\sphinxhyphen{}2}
\item\relax\sphinxstyleindexentry{definition\sphinxhyphen{}2}\sphinxstyleindexextra{ch/principles\sphinxhyphen{}gibbs\sphinxhyphen{}phase\sphinxhyphen{}rule}\sphinxstyleindexpageref{ch/principles-gibbs-phase-rule:\detokenize{definition-2}}
\indexspace
\bigletter{definition\sphinxhyphen{}3}
\item\relax\sphinxstyleindexentry{definition\sphinxhyphen{}3}\sphinxstyleindexextra{ch/principles\sphinxhyphen{}gibbs\sphinxhyphen{}phase\sphinxhyphen{}rule}\sphinxstyleindexpageref{ch/principles-gibbs-phase-rule:\detokenize{definition-3}}
\indexspace
\bigletter{example\sphinxhyphen{}1}
\item\relax\sphinxstyleindexentry{example\sphinxhyphen{}1}\sphinxstyleindexextra{ch/principles\sphinxhyphen{}gibbs\sphinxhyphen{}phase\sphinxhyphen{}rule}\sphinxstyleindexpageref{ch/principles-gibbs-phase-rule:\detokenize{example-1}}
\indexspace
\bigletter{example\sphinxhyphen{}10}
\item\relax\sphinxstyleindexentry{example\sphinxhyphen{}10}\sphinxstyleindexextra{ch/principles\sphinxhyphen{}gibbs\sphinxhyphen{}phase\sphinxhyphen{}rule}\sphinxstyleindexpageref{ch/principles-gibbs-phase-rule:\detokenize{example-10}}
\indexspace
\bigletter{example\sphinxhyphen{}11}
\item\relax\sphinxstyleindexentry{example\sphinxhyphen{}11}\sphinxstyleindexextra{ch/principles\sphinxhyphen{}gibbs\sphinxhyphen{}phase\sphinxhyphen{}rule}\sphinxstyleindexpageref{ch/principles-gibbs-phase-rule:\detokenize{example-11}}
\indexspace
\bigletter{example\sphinxhyphen{}12}
\item\relax\sphinxstyleindexentry{example\sphinxhyphen{}12}\sphinxstyleindexextra{ch/principles\sphinxhyphen{}gibbs\sphinxhyphen{}phase\sphinxhyphen{}rule}\sphinxstyleindexpageref{ch/principles-gibbs-phase-rule:\detokenize{example-12}}
\indexspace
\bigletter{example\sphinxhyphen{}13}
\item\relax\sphinxstyleindexentry{example\sphinxhyphen{}13}\sphinxstyleindexextra{ch/principles\sphinxhyphen{}gibbs\sphinxhyphen{}phase\sphinxhyphen{}rule}\sphinxstyleindexpageref{ch/principles-gibbs-phase-rule:\detokenize{example-13}}
\indexspace
\bigletter{example\sphinxhyphen{}14}
\item\relax\sphinxstyleindexentry{example\sphinxhyphen{}14}\sphinxstyleindexextra{ch/principles\sphinxhyphen{}gibbs\sphinxhyphen{}phase\sphinxhyphen{}rule}\sphinxstyleindexpageref{ch/principles-gibbs-phase-rule:\detokenize{example-14}}
\indexspace
\bigletter{example\sphinxhyphen{}15}
\item\relax\sphinxstyleindexentry{example\sphinxhyphen{}15}\sphinxstyleindexextra{ch/principles\sphinxhyphen{}gibbs\sphinxhyphen{}phase\sphinxhyphen{}rule}\sphinxstyleindexpageref{ch/principles-gibbs-phase-rule:\detokenize{example-15}}
\indexspace
\bigletter{example\sphinxhyphen{}16}
\item\relax\sphinxstyleindexentry{example\sphinxhyphen{}16}\sphinxstyleindexextra{ch/principles\sphinxhyphen{}gibbs\sphinxhyphen{}phase\sphinxhyphen{}rule}\sphinxstyleindexpageref{ch/principles-gibbs-phase-rule:\detokenize{example-16}}
\indexspace
\bigletter{example\sphinxhyphen{}17}
\item\relax\sphinxstyleindexentry{example\sphinxhyphen{}17}\sphinxstyleindexextra{ch/principles\sphinxhyphen{}gibbs\sphinxhyphen{}phase\sphinxhyphen{}rule}\sphinxstyleindexpageref{ch/principles-gibbs-phase-rule:\detokenize{example-17}}
\indexspace
\bigletter{example\sphinxhyphen{}18}
\item\relax\sphinxstyleindexentry{example\sphinxhyphen{}18}\sphinxstyleindexextra{ch/principles\sphinxhyphen{}gibbs\sphinxhyphen{}phase\sphinxhyphen{}rule}\sphinxstyleindexpageref{ch/principles-gibbs-phase-rule:\detokenize{example-18}}
\indexspace
\bigletter{example\sphinxhyphen{}5}
\item\relax\sphinxstyleindexentry{example\sphinxhyphen{}5}\sphinxstyleindexextra{ch/principles\sphinxhyphen{}gibbs\sphinxhyphen{}phase\sphinxhyphen{}rule}\sphinxstyleindexpageref{ch/principles-gibbs-phase-rule:\detokenize{example-5}}
\indexspace
\bigletter{example\sphinxhyphen{}6}
\item\relax\sphinxstyleindexentry{example\sphinxhyphen{}6}\sphinxstyleindexextra{ch/principles\sphinxhyphen{}gibbs\sphinxhyphen{}phase\sphinxhyphen{}rule}\sphinxstyleindexpageref{ch/principles-gibbs-phase-rule:\detokenize{example-6}}
\indexspace
\bigletter{example\sphinxhyphen{}7}
\item\relax\sphinxstyleindexentry{example\sphinxhyphen{}7}\sphinxstyleindexextra{ch/principles\sphinxhyphen{}gibbs\sphinxhyphen{}phase\sphinxhyphen{}rule}\sphinxstyleindexpageref{ch/principles-gibbs-phase-rule:\detokenize{example-7}}
\indexspace
\bigletter{example\sphinxhyphen{}8}
\item\relax\sphinxstyleindexentry{example\sphinxhyphen{}8}\sphinxstyleindexextra{ch/principles\sphinxhyphen{}gibbs\sphinxhyphen{}phase\sphinxhyphen{}rule}\sphinxstyleindexpageref{ch/principles-gibbs-phase-rule:\detokenize{example-8}}
\indexspace
\bigletter{example\sphinxhyphen{}9}
\item\relax\sphinxstyleindexentry{example\sphinxhyphen{}9}\sphinxstyleindexextra{ch/principles\sphinxhyphen{}gibbs\sphinxhyphen{}phase\sphinxhyphen{}rule}\sphinxstyleindexpageref{ch/principles-gibbs-phase-rule:\detokenize{example-9}}
\indexspace
\bigletter{proposition\sphinxhyphen{}4}
\item\relax\sphinxstyleindexentry{proposition\sphinxhyphen{}4}\sphinxstyleindexextra{ch/principles\sphinxhyphen{}gibbs\sphinxhyphen{}phase\sphinxhyphen{}rule}\sphinxstyleindexpageref{ch/principles-gibbs-phase-rule:\detokenize{proposition-4}}
\indexspace
\bigletter{theorem\sphinxhyphen{}0}
\item\relax\sphinxstyleindexentry{theorem\sphinxhyphen{}0}\sphinxstyleindexextra{ch/potentials}\sphinxstyleindexpageref{ch/potentials:\detokenize{theorem-0}}
\indexspace
\bigletter{thermodynamics:principles:open:ex:comb\sphinxhyphen{}chamber}
\item\relax\sphinxstyleindexentry{thermodynamics:principles:open:ex:comb\sphinxhyphen{}chamber}\sphinxstyleindexextra{ch/principles\sphinxhyphen{}open}\sphinxstyleindexpageref{ch/principles-open:\detokenize{thermodynamics:principles:open:ex:comb-chamber}}
\indexspace
\bigletter{thermodynamics:principles:open:ex:compressor}
\item\relax\sphinxstyleindexentry{thermodynamics:principles:open:ex:compressor}\sphinxstyleindexextra{ch/principles\sphinxhyphen{}open}\sphinxstyleindexpageref{ch/principles-open:\detokenize{thermodynamics:principles:open:ex:compressor}}
\indexspace
\bigletter{thermodynamics:principles:open:ex:turbine}
\item\relax\sphinxstyleindexentry{thermodynamics:principles:open:ex:turbine}\sphinxstyleindexextra{ch/principles\sphinxhyphen{}open}\sphinxstyleindexpageref{ch/principles-open:\detokenize{thermodynamics:principles:open:ex:turbine}}
\end{sphinxtheindex}

\renewcommand{\indexname}{Index}
\printindex
\end{document}