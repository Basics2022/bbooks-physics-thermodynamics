%% Generated by Sphinx.
\def\sphinxdocclass{jupyterBook}
\documentclass[letterpaper,10pt,english]{jupyterBook}
\ifdefined\pdfpxdimen
   \let\sphinxpxdimen\pdfpxdimen\else\newdimen\sphinxpxdimen
\fi \sphinxpxdimen=.75bp\relax
\ifdefined\pdfimageresolution
    \pdfimageresolution= \numexpr \dimexpr1in\relax/\sphinxpxdimen\relax
\fi
%% let collapsible pdf bookmarks panel have high depth per default
\PassOptionsToPackage{bookmarksdepth=5}{hyperref}
%% turn off hyperref patch of \index as sphinx.xdy xindy module takes care of
%% suitable \hyperpage mark-up, working around hyperref-xindy incompatibility
\PassOptionsToPackage{hyperindex=false}{hyperref}
%% memoir class requires extra handling
\makeatletter\@ifclassloaded{memoir}
{\ifdefined\memhyperindexfalse\memhyperindexfalse\fi}{}\makeatother

\PassOptionsToPackage{warn}{textcomp}

\catcode`^^^^00a0\active\protected\def^^^^00a0{\leavevmode\nobreak\ }
\usepackage{cmap}
\usepackage{fontspec}
\defaultfontfeatures[\rmfamily,\sffamily,\ttfamily]{}
\usepackage{amsmath,amssymb,amstext}
\usepackage{polyglossia}
\setmainlanguage{english}



\setmainfont{FreeSerif}[
  Extension      = .otf,
  UprightFont    = *,
  ItalicFont     = *Italic,
  BoldFont       = *Bold,
  BoldItalicFont = *BoldItalic
]
\setsansfont{FreeSans}[
  Extension      = .otf,
  UprightFont    = *,
  ItalicFont     = *Oblique,
  BoldFont       = *Bold,
  BoldItalicFont = *BoldOblique,
]
\setmonofont{FreeMono}[
  Extension      = .otf,
  UprightFont    = *,
  ItalicFont     = *Oblique,
  BoldFont       = *Bold,
  BoldItalicFont = *BoldOblique,
]



\usepackage[Bjarne]{fncychap}
\usepackage[,numfigreset=1,mathnumfig]{sphinx}

\fvset{fontsize=\small}
\usepackage{geometry}


% Include hyperref last.
\usepackage{hyperref}
% Fix anchor placement for figures with captions.
\usepackage{hypcap}% it must be loaded after hyperref.
% Set up styles of URL: it should be placed after hyperref.
\urlstyle{same}

\addto\captionsenglish{\renewcommand{\contentsname}{Termodinamica}}

\usepackage{sphinxmessages}



        % Start of preamble defined in sphinx-jupyterbook-latex %
         \usepackage[Latin,Greek]{ucharclasses}
        \usepackage{unicode-math}
        % fixing title of the toc
        \addto\captionsenglish{\renewcommand{\contentsname}{Contents}}
        \hypersetup{
            pdfencoding=auto,
            psdextra
        }
        % End of preamble defined in sphinx-jupyterbook-latex %
        

\title{Termodinamica e fisica tecnica}
\date{Dec 30, 2024}
\release{}
\author{basics}
\newcommand{\sphinxlogo}{\vbox{}}
\renewcommand{\releasename}{}
\makeindex
\begin{document}

\pagestyle{empty}
\sphinxmaketitle
\pagestyle{plain}
\sphinxtableofcontents
\pagestyle{normal}
\phantomsection\label{\detokenize{intro::doc}}


\sphinxAtStartPar
\sphinxstylestrong{Introduzione.}
\begin{itemize}
\item {} 
\sphinxAtStartPar
la termodinamica si occupa della forme di energia, e delle sue variazioni attraverso scambi di lavoro e calore;

\item {} 
\sphinxAtStartPar
la termodinamica classica fornisce una descrizione macroscopica di sistemi complessi, formati da un numero enorme di componenti elementari (atomi e molecole), per motivi di convenienza: la descrizione macroscopica media comporta una perdita di informazioni sul sistema, e la necessità di una grandezza legata ad essa che indichi il verso naturale di alcune trasformazioni, l’entropia;

\item {} 
\sphinxAtStartPar
in equilibrio \sphinxstylestrong{todo}: in molti casi di interesse comune, l’\sphinxstylestrong{equilibrio} termodinamico \sphinxstylestrong{locale} viene raggiunto in intervalli di tempo che hanno durata caratteristica molto inferiore ai tempi caratteristici del sistema macrroscopico: le leggi della termodinamica vengono quindi applicate anche nello studio della meccanica dei solidi e dei fluidi;

\end{itemize}

\sphinxAtStartPar
\sphinxstylestrong{Argomenti.}
\begin{itemize}
\item {} 
\sphinxAtStartPar
Principi della termodinamica
\begin{itemize}
\item {} 
\sphinxAtStartPar
Modello matematico: funzioni di stato,…

\item {} 
\sphinxAtStartPar
Dall’esperienza ai principi fisici, alla base del modello

\item {} 
\sphinxAtStartPar
Principi per sistemi chiusi e per sistemi aperti: facendo riferiemnto al \sphinxstyleemphasis{thm del trasporto di Reynolds}

\item {} 
\sphinxAtStartPar
grandezze intensive, estensive, sepcifiche,…

\end{itemize}

\item {} 
\sphinxAtStartPar
Potenziali termodinamici e coefficienti termodinamici
\begin{itemize}
\item {} 
\sphinxAtStartPar
potenziali termodinamici, relazioni di Maxwell

\item {} 
\sphinxAtStartPar
coefficienti termodinamici: dilatazione, comprimibilità

\item {} 
\sphinxAtStartPar
miscellanea nell’uso delle derivate parziali di funzioni multi\sphinxhyphen{}variabile in termodinamica

\end{itemize}

\item {} 
\sphinxAtStartPar
Stati della materia: equazioni di stato e equazioni costitutive
\begin{itemize}
\item {} 
\sphinxAtStartPar
fluidi:
\begin{itemize}
\item {} 
\sphinxAtStartPar
gas

\item {} 
\sphinxAtStartPar
liquidi

\end{itemize}

\item {} 
\sphinxAtStartPar
solidi

\item {} 
\sphinxAtStartPar
trasformazioni di fase

\item {} 
\sphinxAtStartPar
altro…

\end{itemize}

\item {} 
\sphinxAtStartPar
Trasformazioni termodinamiche nei gas e macchine a fluido
\begin{itemize}
\item {} 
\sphinxAtStartPar
trasformazioni nei piani termodinamici

\item {} 
\sphinxAtStartPar
macchine termiche ideali e non: Carnot, efficienza massima, cicli reali, formulazioni di Kelvin e Plank del secondo principio della termodinamica; disuguaglianza di Clausis

\item {} 
\sphinxAtStartPar
cicli termodinamici: Otto, Diesel, Rankine, Joule\sphinxhyphen{}Brayton,…

\end{itemize}

\item {} 
\sphinxAtStartPar
Meccanismi di trasmissione del calore:
\begin{itemize}
\item {} 
\sphinxAtStartPar
conduzione

\item {} 
\sphinxAtStartPar
convezione: naturale e forzata

\item {} 
\sphinxAtStartPar
irraggiamento

\end{itemize}

\end{itemize}

\sphinxAtStartPar
\sphinxstylestrong{Extra.}
\begin{itemize}
\item {} 
\sphinxAtStartPar
Formulazione assiomatica della termodinamica

\item {} 
\sphinxAtStartPar
Cenni di meccanica statistica

\end{itemize}

\sphinxstepscope

\begin{sphinxuseclass}{sd-container-fluid}
\begin{sphinxuseclass}{sd-sphinx-override}
\begin{sphinxuseclass}{sd-p-0}
\begin{sphinxuseclass}{sd-mt-2}
\begin{sphinxuseclass}{sd-mb-4}
\begin{sphinxuseclass}{sd-row}
\begin{sphinxuseclass}{sd-row-cols-2}
\begin{sphinxuseclass}{sd-gx-2}
\begin{sphinxuseclass}{sd-gy-1}
\begin{sphinxuseclass}{sd-col}
\begin{sphinxuseclass}{sd-d-flex-row}
\begin{sphinxuseclass}{sd-align-minor-center}
\begin{sphinxuseclass}{sd-container-fluid}
\begin{sphinxuseclass}{sd-sphinx-override}
\begin{sphinxuseclass}{sd-row}
\begin{sphinxuseclass}{sd-row-cols-2}
\begin{sphinxuseclass}{sd-row-cols-xs-2}
\begin{sphinxuseclass}{sd-row-cols-sm-3}
\begin{sphinxuseclass}{sd-row-cols-md-3}
\begin{sphinxuseclass}{sd-row-cols-lg-3}
\begin{sphinxuseclass}{sd-gx-3}
\begin{sphinxuseclass}{sd-gy-1}
\begin{sphinxuseclass}{sd-col}
\begin{sphinxuseclass}{sd-col-auto}
\begin{sphinxuseclass}{sd-d-flex-row}
\begin{sphinxuseclass}{sd-align-minor-center}
\sphinxAtStartPar
basics

\end{sphinxuseclass}
\end{sphinxuseclass}
\end{sphinxuseclass}
\end{sphinxuseclass}
\begin{sphinxuseclass}{sd-col}
\begin{sphinxuseclass}{sd-col-auto}
\begin{sphinxuseclass}{sd-d-flex-row}
\begin{sphinxuseclass}{sd-align-minor-center}
\sphinxAtStartPar
Dec 30, 2024

\end{sphinxuseclass}
\end{sphinxuseclass}
\end{sphinxuseclass}
\end{sphinxuseclass}
\begin{sphinxuseclass}{sd-col}
\begin{sphinxuseclass}{sd-col-auto}
\begin{sphinxuseclass}{sd-d-flex-row}
\begin{sphinxuseclass}{sd-align-minor-center}
\sphinxAtStartPar
0 min read

\end{sphinxuseclass}
\end{sphinxuseclass}
\end{sphinxuseclass}
\end{sphinxuseclass}
\end{sphinxuseclass}
\end{sphinxuseclass}
\end{sphinxuseclass}
\end{sphinxuseclass}
\end{sphinxuseclass}
\end{sphinxuseclass}
\end{sphinxuseclass}
\end{sphinxuseclass}
\end{sphinxuseclass}
\end{sphinxuseclass}
\end{sphinxuseclass}
\end{sphinxuseclass}
\end{sphinxuseclass}
\end{sphinxuseclass}
\end{sphinxuseclass}
\end{sphinxuseclass}
\end{sphinxuseclass}
\end{sphinxuseclass}
\end{sphinxuseclass}
\end{sphinxuseclass}
\end{sphinxuseclass}
\end{sphinxuseclass}

\chapter{Storia della termodinamica}
\label{\detokenize{ch/history:storia-della-termodinamica}}\label{\detokenize{ch/history:classical-thermodynamics-history}}\label{\detokenize{ch/history::doc}}


\sphinxstepscope

\begin{sphinxuseclass}{sd-container-fluid}
\begin{sphinxuseclass}{sd-sphinx-override}
\begin{sphinxuseclass}{sd-p-0}
\begin{sphinxuseclass}{sd-mt-2}
\begin{sphinxuseclass}{sd-mb-4}
\begin{sphinxuseclass}{sd-row}
\begin{sphinxuseclass}{sd-row-cols-2}
\begin{sphinxuseclass}{sd-gx-2}
\begin{sphinxuseclass}{sd-gy-1}
\begin{sphinxuseclass}{sd-col}
\begin{sphinxuseclass}{sd-d-flex-row}
\begin{sphinxuseclass}{sd-align-minor-center}
\begin{sphinxuseclass}{sd-container-fluid}
\begin{sphinxuseclass}{sd-sphinx-override}
\begin{sphinxuseclass}{sd-row}
\begin{sphinxuseclass}{sd-row-cols-2}
\begin{sphinxuseclass}{sd-row-cols-xs-2}
\begin{sphinxuseclass}{sd-row-cols-sm-3}
\begin{sphinxuseclass}{sd-row-cols-md-3}
\begin{sphinxuseclass}{sd-row-cols-lg-3}
\begin{sphinxuseclass}{sd-gx-3}
\begin{sphinxuseclass}{sd-gy-1}
\begin{sphinxuseclass}{sd-col}
\begin{sphinxuseclass}{sd-col-auto}
\begin{sphinxuseclass}{sd-d-flex-row}
\begin{sphinxuseclass}{sd-align-minor-center}
\sphinxAtStartPar
basics

\end{sphinxuseclass}
\end{sphinxuseclass}
\end{sphinxuseclass}
\end{sphinxuseclass}
\begin{sphinxuseclass}{sd-col}
\begin{sphinxuseclass}{sd-col-auto}
\begin{sphinxuseclass}{sd-d-flex-row}
\begin{sphinxuseclass}{sd-align-minor-center}
\sphinxAtStartPar
Dec 30, 2024

\end{sphinxuseclass}
\end{sphinxuseclass}
\end{sphinxuseclass}
\end{sphinxuseclass}
\begin{sphinxuseclass}{sd-col}
\begin{sphinxuseclass}{sd-col-auto}
\begin{sphinxuseclass}{sd-d-flex-row}
\begin{sphinxuseclass}{sd-align-minor-center}
\sphinxAtStartPar
2 min read

\end{sphinxuseclass}
\end{sphinxuseclass}
\end{sphinxuseclass}
\end{sphinxuseclass}
\end{sphinxuseclass}
\end{sphinxuseclass}
\end{sphinxuseclass}
\end{sphinxuseclass}
\end{sphinxuseclass}
\end{sphinxuseclass}
\end{sphinxuseclass}
\end{sphinxuseclass}
\end{sphinxuseclass}
\end{sphinxuseclass}
\end{sphinxuseclass}
\end{sphinxuseclass}
\end{sphinxuseclass}
\end{sphinxuseclass}
\end{sphinxuseclass}
\end{sphinxuseclass}
\end{sphinxuseclass}
\end{sphinxuseclass}
\end{sphinxuseclass}
\end{sphinxuseclass}
\end{sphinxuseclass}
\end{sphinxuseclass}

\chapter{Princìpi della termodinamica}
\label{\detokenize{ch/principles:principi-della-termodinamica}}\label{\detokenize{ch/principles:classical-thermodynamics-principles}}\label{\detokenize{ch/principles::doc}}
\sphinxAtStartPar
\sphinxstylestrong{Primo principio della termodinamica.} Il primo principio della termodinamica è il bilancio di energia totale di un sistema
\begin{equation*}
\begin{split}\dot{E}^{tot} = P^{e} + \dot{Q}^{e}\end{split}
\end{equation*}
\sphinxAtStartPar
L’energia totale di un sistema può essere scritta come la somma dell’energia cinetica macroscopica \(K\) e l’energia interna \(E\), una rappresentazione macroscopica dell’energia cinetica delle componenti microscopiche del sistema, attorno ai valori medi macroscopici locali,
\begin{equation*}
\begin{split}E^{tot} = K + E \ .\end{split}
\end{equation*}
\sphinxAtStartPar
Usando il teorema dell’energia cinetica \sphinxstyleemphasis{macroscopica} \sphinxstylestrong{{[}REF{]}} noto dalla meccanica si può ricavare
\begin{equation*}
\begin{split}\begin{aligned}
  \dot{E}^{tot} & = P^{e} + \dot{Q}^e \\
  \dot{K}       & = P^{tot} = P^{e} + P^{i} \\
  \dot{E}       & = \dot{Q}^e - P^{i} 
\end{aligned}\end{split}
\end{equation*}
\sphinxAtStartPar
Il bilancio dell’energia interna può essere scritto in forma incrementale come
\begin{equation*}
\begin{split}d E = d\hspace{-0.08pt}\bar{}\hspace{0.1pt} Q^e - d\hspace{-0.08pt}\bar{}\hspace{0.1pt} L^i \ ,\end{split}
\end{equation*}
\sphinxAtStartPar
mettendo in evidenza con la notazione che l’energia interna è una variabile di stato del sistema, a differenza degli scambi di lavoro o di calore.

\sphinxAtStartPar
Si assume che l’energia interna del sistema può essere scritta come funzione di variabili di stato cinematiche locali \(\mathbf{x}\) (che non possono contribuire all’energia cinetica macroscopica del sistema) e almeno un’altra variabile di stato additiva, qui chiamata \(S\).
\begin{equation*}
\begin{split}E(\mathbf{x}, S)\end{split}
\end{equation*}
\sphinxAtStartPar
Il suo differenziale può essere scritto come
\begin{equation*}
\begin{split}d E = \dfrac{\partial E}{\partial \mathbf{x}}\Big|_{S} \cdot d \mathbf{x} + \dfrac{\partial E}{\partial S}\Big|_{\mathbf{x}} \cdot d \mathbf{S}\end{split}
\end{equation*}
\sphinxAtStartPar
Il bilancio di energia interna in forma incrementale può essere riscritto come
\begin{equation*}
\begin{split}\begin{aligned}
d E & = - d\hspace{-0.08pt}\bar{}\hspace{0.1pt} L^i + d\hspace{-0.08pt}\bar{}\hspace{0.1pt} Q^e =  \\
    & = - \delta L^{i,rev} + d\hspace{-0.08pt}\bar{}\hspace{0.1pt}^+ D + d\hspace{-0.08pt}\bar{}\hspace{0.1pt} Q^e  \\
\end{aligned}\end{split}
\end{equation*}
\sphinxAtStartPar
avendo separato nel lavoro interno il contributo delle azioni reversibili e non reversibili. In accordo con l’esperienza, si assume che il contributo al lavoro interno delle forze non reversibili è sempre positivo, \(d\hspace{-0.08pt}\bar{}\hspace{0.1pt}^+ D \ge 0\): questo contributo viene chiamato \sphinxstylestrong{dissipazione}.

\sphinxAtStartPar
Poiché sia il differenziale dell’energia e il contributo reversibile del lavoro interno sono dei termini reversibili, allora anche la somma di dissipazione e flusso di calore deve essere un contributo reversibile, indicato con \(\delta U\).

\sphinxAtStartPar
Dal confronto delle due espressioni del differenziale dell’energia interna, avendo riconosciuto il contributo reversibile del lavoro interno nel primo termine, si può scrivere
\begin{equation*}
\begin{split}
  \delta L^{i,rev} = - \dfrac{\partial E}{\partial \mathbf{x}} \Big|_{S} d \mathbf{x}  \qquad , \qquad 
  \delta U         = \ \dfrac{\partial E}{\partial S} \Big|_{\mathbf{x}} d S 
\end{split}
\end{equation*}
\sphinxAtStartPar
\sphinxstylestrong{Terzo principio della termodinamica: definizione e segno della temperatura.}
La temperatura viene definita come la variabile parziale dell’energia interna del sistema rispetto alla variabile di stato \(S\),
\begin{equation*}
\begin{split}T := \dfrac{\partial E}{\partial S}\Big|_S{} \ .\end{split}
\end{equation*}
\sphinxAtStartPar
Il terzo principio della termodinamica postula che la temperatura sia sempre positiva.
\begin{equation*}
\begin{split}T := \dfrac{\partial E}{\partial S}\Big|_S{} > 0 \ .\end{split}
\end{equation*}
\sphinxAtStartPar
\sphinxstylestrong{Secondo principio della termodinamica: enunciato di Clausius.} L’enunciato di Clausis del secondo principio della termodinamica può essere formulato in accordo con l’evidenza che la dissipazione sia sempre non negativa. Infatti
\begin{equation*}
\begin{split}T dS =: \partial U = \underbrace{d\hspace{-0.08pt}\bar{}\hspace{0.1pt}^+ D}_{\ge 0} + d\hspace{-0.08pt}\bar{}\hspace{0.1pt} Q^e \ge  d\hspace{-0.08pt}\bar{}\hspace{0.1pt} Q^e \ ,\end{split}
\end{equation*}
\sphinxAtStartPar
e quindi
\begin{equation*}
\begin{split}dS \ge \dfrac{d\hspace{-0.08pt}\bar{}\hspace{0.1pt} Q^e }{T} \ .\end{split}
\end{equation*}
\sphinxAtStartPar
\sphinxstylestrong{Secondo principio della termodinamica per sistemi complessi e direzione dei trasferimenti di calore.}

\sphinxAtStartPar
L’esperienza evidenzia che il trasferimento di calore avviene dai corpi a temperatura maggiore a corpi a temperatura minore, ossia il flusso di calore fa aumentare l’energia interna del corpo freddo e diminuire quella del corpo caldo.

\sphinxAtStartPar
Nello scabio di calore tra due corpi \(i\), \(j\) a temperature \(T_i\), \(T_j\), questa evidenza sperimentale può essere scritta come
\begin{equation*}
\begin{split}d\hspace{-0.08pt}\bar{}\hspace{0.1pt} Q_{ji} \gtreqless 0 \quad \text{se} \quad T_i \gtreqless T_j \qquad \rightarrow \qquad \dfrac{d\hspace{-0.08pt}\bar{}\hspace{0.1pt} Q_{ji}}{T_j} + \dfrac{d\hspace{-0.08pt}\bar{}\hspace{0.1pt} Q_{ij}}{T_i} \ge 0 \ ,\end{split}
\end{equation*}
\sphinxAtStartPar
dove \(d\hspace{-0.08pt}\bar{}\hspace{0.1pt} Q_{ji} = - d\hspace{-0.08pt}\bar{}\hspace{0.1pt} Q_{ij}\) è il flusso di calore dal corpo \(i\) al corpo \(j\), positivo se fa aumentare l’energia interna di \(j\) e diminuire quella di \(i\).
\begin{equation*}
\begin{split}d S_i = \underbrace{\dfrac{d\hspace{-0.08pt}\bar{}\hspace{0.1pt}  D_i}{T_i}}_{\ge 0} + \dfrac{d\hspace{-0.08pt}\bar{}\hspace{0.1pt}  Q^{e,i}}{T_i} + \sum_{j \ne i} \dfrac{d\hspace{-0.08pt}\bar{}\hspace{0.1pt}  Q_{ij}}{T_i}\end{split}
\end{equation*}
\sphinxAtStartPar
Poichè \(S\) è una variabile estensiva, il valore associato all’intero sistema è uguale alla somma dei valori delle sue parti, \(S = \sum_i S_i\). E’ quindi possibile ricavare una relazione per l’intero sistema sommando i contributi dovuti a tutte le sue parti.
\begin{equation*}
\begin{split}\begin{aligned}
  d S & = \sum_i d S_i = \\
      & = \sum_i \sum_i \dfrac{d\hspace{-0.08pt}\bar{}\hspace{0.1pt}  Q^{e,i}}{T_i} + \sum_i \sum_{j \ne i} \dfrac{d\hspace{-0.08pt}\bar{}\hspace{0.1pt}  Q_{ij}}{T_i} + \underbrace{\dfrac{d\hspace{-0.08pt}\bar{}\hspace{0.1pt}  D_i}{T_i}}_{\ge 0} = \\
      & \ge \sum_i \dfrac{d\hspace{-0.08pt}\bar{}\hspace{0.1pt}  Q^{e,i}}{T_i} + \sum_{\{i, j\}} \underbrace{\Big( \dfrac{d\hspace{-0.08pt}\bar{}\hspace{0.1pt}  Q_{ij}}{T_i} + \dfrac{d\hspace{-0.08pt}\bar{}\hspace{0.1pt}  Q_{ji}}{T_j} \Big)}_{ \ge 0 }  = \\
      & \ge \sum_i \dfrac{d\hspace{-0.08pt}\bar{}\hspace{0.1pt}  Q^{e,i}}{T_i} \ .
\end{aligned}\end{split}
\end{equation*}
\sphinxstepscope

\begin{sphinxuseclass}{sd-container-fluid}
\begin{sphinxuseclass}{sd-sphinx-override}
\begin{sphinxuseclass}{sd-p-0}
\begin{sphinxuseclass}{sd-mt-2}
\begin{sphinxuseclass}{sd-mb-4}
\begin{sphinxuseclass}{sd-row}
\begin{sphinxuseclass}{sd-row-cols-2}
\begin{sphinxuseclass}{sd-gx-2}
\begin{sphinxuseclass}{sd-gy-1}
\begin{sphinxuseclass}{sd-col}
\begin{sphinxuseclass}{sd-d-flex-row}
\begin{sphinxuseclass}{sd-align-minor-center}
\begin{sphinxuseclass}{sd-container-fluid}
\begin{sphinxuseclass}{sd-sphinx-override}
\begin{sphinxuseclass}{sd-row}
\begin{sphinxuseclass}{sd-row-cols-2}
\begin{sphinxuseclass}{sd-row-cols-xs-2}
\begin{sphinxuseclass}{sd-row-cols-sm-3}
\begin{sphinxuseclass}{sd-row-cols-md-3}
\begin{sphinxuseclass}{sd-row-cols-lg-3}
\begin{sphinxuseclass}{sd-gx-3}
\begin{sphinxuseclass}{sd-gy-1}
\begin{sphinxuseclass}{sd-col}
\begin{sphinxuseclass}{sd-col-auto}
\begin{sphinxuseclass}{sd-d-flex-row}
\begin{sphinxuseclass}{sd-align-minor-center}
\sphinxAtStartPar
basics

\end{sphinxuseclass}
\end{sphinxuseclass}
\end{sphinxuseclass}
\end{sphinxuseclass}
\begin{sphinxuseclass}{sd-col}
\begin{sphinxuseclass}{sd-col-auto}
\begin{sphinxuseclass}{sd-d-flex-row}
\begin{sphinxuseclass}{sd-align-minor-center}
\sphinxAtStartPar
Dec 30, 2024

\end{sphinxuseclass}
\end{sphinxuseclass}
\end{sphinxuseclass}
\end{sphinxuseclass}
\begin{sphinxuseclass}{sd-col}
\begin{sphinxuseclass}{sd-col-auto}
\begin{sphinxuseclass}{sd-d-flex-row}
\begin{sphinxuseclass}{sd-align-minor-center}
\sphinxAtStartPar
0 min read

\end{sphinxuseclass}
\end{sphinxuseclass}
\end{sphinxuseclass}
\end{sphinxuseclass}
\end{sphinxuseclass}
\end{sphinxuseclass}
\end{sphinxuseclass}
\end{sphinxuseclass}
\end{sphinxuseclass}
\end{sphinxuseclass}
\end{sphinxuseclass}
\end{sphinxuseclass}
\end{sphinxuseclass}
\end{sphinxuseclass}
\end{sphinxuseclass}
\end{sphinxuseclass}
\end{sphinxuseclass}
\end{sphinxuseclass}
\end{sphinxuseclass}
\end{sphinxuseclass}
\end{sphinxuseclass}
\end{sphinxuseclass}
\end{sphinxuseclass}
\end{sphinxuseclass}
\end{sphinxuseclass}
\end{sphinxuseclass}

\chapter{Potenziali termodinamici}
\label{\detokenize{ch/potentials:potenziali-termodinamici}}\label{\detokenize{ch/potentials:classical-thermodynamics-potentials}}\label{\detokenize{ch/potentials::doc}}
\sphinxAtStartPar
Partendo dall’espressione incrementale del primo principio della termodinamica è possivile definire altre funzioni delle variabili di stato (\sphinxstylestrong{TODO} \sphinxstyleemphasis{Ruolo della trasformata di Legendre}), chiamati \sphinxstylestrong{potenziali termodinamici}

\sphinxAtStartPar
\sphinxstylestrong{Energia interna.} \(E(\mathbf{x}, S)\)
\begin{equation*}
\begin{split}d E = - \mathbf{F} \cdot d \mathbf{x} + T dS\end{split}
\end{equation*}
\sphinxAtStartPar
\sphinxstylestrong{Energia libera di Helmholtz.} \(F(\mathbf{x}, T) := E - T S\)
\begin{equation*}
\begin{split}\begin{aligned}
d F & = d E - T dS - S dT = \\
    & = - \mathbf{F} \cdot d \mathbf{x} + T dS - T dS - S dT = \\
    & = - \mathbf{F} \cdot d \mathbf{x} - S dT 
\end{aligned}\end{split}
\end{equation*}
\sphinxAtStartPar
\sphinxstylestrong{Entalpia.} \(H(\mathbf{F}, S) := E + \mathbf{F} \cdot \mathbf{x}\)
\begin{equation*}
\begin{split}\begin{aligned}
d H & = d E + \mathbf{F} \cdot d \mathbf{x} + \mathbf{x} \cdot d \mathbf{F} = \\
    & = - \mathbf{F} \cdot d \mathbf{x} + T dS + \mathbf{F} \cdot d \mathbf{x} + \mathbf{x} \cdot d \mathbf{F}= \\
    & = \ \mathbf{x} \cdot d \mathbf{F} + T dS 
\end{aligned}\end{split}
\end{equation*}
\sphinxAtStartPar
\sphinxstylestrong{Energia libera di Gibbs.} \(G(\mathbf{F}, T) := H - T S\)
\begin{equation*}
\begin{split}\begin{aligned}
d G & = d H - T dS - S dT = \\
    & = \ \mathbf{x} \cdot d \mathbf{F} + T dS - T dS - S dT = \\
    & = \ \mathbf{x} \cdot d \mathbf{F} - S dT 
\end{aligned}\end{split}
\end{equation*}
\sphinxAtStartPar
\sphinxstylestrong{Derivate parziali dei potenziali come definizione di variabili termodinamiche.} Osservando i differenziali dei potenziali termodinamici, si possono riconoscere che le variabili termodinamiche \(T\), \(S\), \(\mathbf{F}\), \(\mathbf{x}\) possono essere scritte come derivate parziali dei potenziali termodinamici,
\begin{equation*}
\begin{split}\begin{aligned}
 T & = \quad \frac{\partial E}{\partial S}\Big|_{\mathbf{x}} & = \quad \frac{\partial H}{\partial S}\Big|_{\mathbf{F}} \\
 S & =     - \frac{\partial F}{\partial T}\Big|_{\mathbf{x}} & =     - \frac{\partial G}{\partial T}\Big|_{\mathbf{F}} \\
 \mathbf{F} & =     - \frac{\partial E}{\partial \mathbf{x}}\Big|_{S} & = - \frac{\partial F}{\partial \mathbf{x}}\Big|_{T} \\
 \mathbf{x} & = \quad \frac{\partial H}{\partial \mathbf{F}}\Big|_{S} & = \quad \frac{\partial G}{\partial \mathbf{F}}\Big|_{T}\\
\end{aligned}\end{split}
\end{equation*}
\sphinxAtStartPar
\sphinxstylestrong{Relazioni di Maxwell.} Applicando il \sphinxstylestrong{teorema di Schwarz} sulle derivate miste ai potenziali termodinamici, si ricavano le relazioni di Maxwell,
\begin{equation*}
\begin{split}\begin{cases}
 \dfrac{\partial T}{\partial x_i} & =   - \dfrac{\partial F_i}{\partial S} \\
 \dfrac{\partial S}{\partial x_i} & = \ \ \dfrac{\partial F_i}{\partial T} \\
 \dfrac{\partial T}{\partial F_i} & = \ \ \dfrac{\partial x_i}{\partial S} \\
 \dfrac{\partial S}{\partial F_i} & =   - \dfrac{\partial x_i}{\partial T} \\
\end{cases}\end{split}
\end{equation*}
\sphinxAtStartPar
e
\begin{equation*}
\begin{split}\begin{cases}
 \dfrac{\partial F_i}{\partial x_j} & =  \dfrac{\partial F_j}{\partial x_i} \\
 \dfrac{\partial x_i}{\partial F_j} & =  \dfrac{\partial x_j}{\partial F_i} \\
\end{cases}\end{split}
\end{equation*}
\sphinxstepscope

\begin{sphinxuseclass}{sd-container-fluid}
\begin{sphinxuseclass}{sd-sphinx-override}
\begin{sphinxuseclass}{sd-p-0}
\begin{sphinxuseclass}{sd-mt-2}
\begin{sphinxuseclass}{sd-mb-4}
\begin{sphinxuseclass}{sd-row}
\begin{sphinxuseclass}{sd-row-cols-2}
\begin{sphinxuseclass}{sd-gx-2}
\begin{sphinxuseclass}{sd-gy-1}
\begin{sphinxuseclass}{sd-col}
\begin{sphinxuseclass}{sd-d-flex-row}
\begin{sphinxuseclass}{sd-align-minor-center}
\begin{sphinxuseclass}{sd-container-fluid}
\begin{sphinxuseclass}{sd-sphinx-override}
\begin{sphinxuseclass}{sd-row}
\begin{sphinxuseclass}{sd-row-cols-2}
\begin{sphinxuseclass}{sd-row-cols-xs-2}
\begin{sphinxuseclass}{sd-row-cols-sm-3}
\begin{sphinxuseclass}{sd-row-cols-md-3}
\begin{sphinxuseclass}{sd-row-cols-lg-3}
\begin{sphinxuseclass}{sd-gx-3}
\begin{sphinxuseclass}{sd-gy-1}
\begin{sphinxuseclass}{sd-col}
\begin{sphinxuseclass}{sd-col-auto}
\begin{sphinxuseclass}{sd-d-flex-row}
\begin{sphinxuseclass}{sd-align-minor-center}
\sphinxAtStartPar
basics

\end{sphinxuseclass}
\end{sphinxuseclass}
\end{sphinxuseclass}
\end{sphinxuseclass}
\begin{sphinxuseclass}{sd-col}
\begin{sphinxuseclass}{sd-col-auto}
\begin{sphinxuseclass}{sd-d-flex-row}
\begin{sphinxuseclass}{sd-align-minor-center}
\sphinxAtStartPar
Dec 30, 2024

\end{sphinxuseclass}
\end{sphinxuseclass}
\end{sphinxuseclass}
\end{sphinxuseclass}
\begin{sphinxuseclass}{sd-col}
\begin{sphinxuseclass}{sd-col-auto}
\begin{sphinxuseclass}{sd-d-flex-row}
\begin{sphinxuseclass}{sd-align-minor-center}
\sphinxAtStartPar
0 min read

\end{sphinxuseclass}
\end{sphinxuseclass}
\end{sphinxuseclass}
\end{sphinxuseclass}
\end{sphinxuseclass}
\end{sphinxuseclass}
\end{sphinxuseclass}
\end{sphinxuseclass}
\end{sphinxuseclass}
\end{sphinxuseclass}
\end{sphinxuseclass}
\end{sphinxuseclass}
\end{sphinxuseclass}
\end{sphinxuseclass}
\end{sphinxuseclass}
\end{sphinxuseclass}
\end{sphinxuseclass}
\end{sphinxuseclass}
\end{sphinxuseclass}
\end{sphinxuseclass}
\end{sphinxuseclass}
\end{sphinxuseclass}
\end{sphinxuseclass}
\end{sphinxuseclass}
\end{sphinxuseclass}
\end{sphinxuseclass}

\chapter{Coefficienti termodinamici}
\label{\detokenize{ch/coefficients:coefficienti-termodinamici}}\label{\detokenize{ch/coefficients:classical-thermodynamics-coefficients}}\label{\detokenize{ch/coefficients::doc}}
\sphinxAtStartPar
\sphinxstylestrong{Calore specifico.}
\begin{equation*}
\begin{split}S_x := T \dfrac{\partial S}{\partial T}\Big|_x\end{split}
\end{equation*}
\sphinxAtStartPar
\sphinxstylestrong{Coefficienti di espansione termica.}
\begin{equation*}
\begin{split}\alpha_x := \dfrac{1}{v} \dfrac{\partial v}{\partial T}\Big|_x = - \dfrac{1}{\rho} \dfrac{\partial \rho}{\partial T}\Big|_x\end{split}
\end{equation*}
\sphinxAtStartPar
\sphinxstylestrong{Coefficienti di comprimibilità.}
\begin{equation*}
\begin{split}\beta_x := - \dfrac{1}{v} \dfrac{\partial v}{\partial P}\Big|_x = \dfrac{1}{\rho} \dfrac{\partial \rho}{\partial P}\Big|_x\end{split}
\end{equation*}


\sphinxstepscope

\begin{sphinxuseclass}{sd-container-fluid}
\begin{sphinxuseclass}{sd-sphinx-override}
\begin{sphinxuseclass}{sd-p-0}
\begin{sphinxuseclass}{sd-mt-2}
\begin{sphinxuseclass}{sd-mb-4}
\begin{sphinxuseclass}{sd-row}
\begin{sphinxuseclass}{sd-row-cols-2}
\begin{sphinxuseclass}{sd-gx-2}
\begin{sphinxuseclass}{sd-gy-1}
\begin{sphinxuseclass}{sd-col}
\begin{sphinxuseclass}{sd-d-flex-row}
\begin{sphinxuseclass}{sd-align-minor-center}
\begin{sphinxuseclass}{sd-container-fluid}
\begin{sphinxuseclass}{sd-sphinx-override}
\begin{sphinxuseclass}{sd-row}
\begin{sphinxuseclass}{sd-row-cols-2}
\begin{sphinxuseclass}{sd-row-cols-xs-2}
\begin{sphinxuseclass}{sd-row-cols-sm-3}
\begin{sphinxuseclass}{sd-row-cols-md-3}
\begin{sphinxuseclass}{sd-row-cols-lg-3}
\begin{sphinxuseclass}{sd-gx-3}
\begin{sphinxuseclass}{sd-gy-1}
\begin{sphinxuseclass}{sd-col}
\begin{sphinxuseclass}{sd-col-auto}
\begin{sphinxuseclass}{sd-d-flex-row}
\begin{sphinxuseclass}{sd-align-minor-center}
\sphinxAtStartPar
basics

\end{sphinxuseclass}
\end{sphinxuseclass}
\end{sphinxuseclass}
\end{sphinxuseclass}
\begin{sphinxuseclass}{sd-col}
\begin{sphinxuseclass}{sd-col-auto}
\begin{sphinxuseclass}{sd-d-flex-row}
\begin{sphinxuseclass}{sd-align-minor-center}
\sphinxAtStartPar
Dec 30, 2024

\end{sphinxuseclass}
\end{sphinxuseclass}
\end{sphinxuseclass}
\end{sphinxuseclass}
\begin{sphinxuseclass}{sd-col}
\begin{sphinxuseclass}{sd-col-auto}
\begin{sphinxuseclass}{sd-d-flex-row}
\begin{sphinxuseclass}{sd-align-minor-center}
\sphinxAtStartPar
0 min read

\end{sphinxuseclass}
\end{sphinxuseclass}
\end{sphinxuseclass}
\end{sphinxuseclass}
\end{sphinxuseclass}
\end{sphinxuseclass}
\end{sphinxuseclass}
\end{sphinxuseclass}
\end{sphinxuseclass}
\end{sphinxuseclass}
\end{sphinxuseclass}
\end{sphinxuseclass}
\end{sphinxuseclass}
\end{sphinxuseclass}
\end{sphinxuseclass}
\end{sphinxuseclass}
\end{sphinxuseclass}
\end{sphinxuseclass}
\end{sphinxuseclass}
\end{sphinxuseclass}
\end{sphinxuseclass}
\end{sphinxuseclass}
\end{sphinxuseclass}
\end{sphinxuseclass}
\end{sphinxuseclass}
\end{sphinxuseclass}

\chapter{Stati della materia e modelli}
\label{\detokenize{ch/media:stati-della-materia-e-modelli}}\label{\detokenize{ch/media:classical-thermodynamics-media}}\label{\detokenize{ch/media::doc}}
\sphinxAtStartPar
\sphinxstylestrong{Stati della materia.}
\begin{itemize}
\item {} 
\sphinxAtStartPar
gas

\item {} 
\sphinxAtStartPar
liquidi

\item {} 
\sphinxAtStartPar
solidi

\item {} 
\sphinxAtStartPar
plasma

\end{itemize}

\sphinxAtStartPar
\sphinxstylestrong{Alcune leggi costitutive.}
\begin{itemize}
\item {} 
\sphinxAtStartPar
solidi elastici:
\begin{itemize}
\item {} 
\sphinxAtStartPar
solidi lineari elastici isotropi

\end{itemize}

\item {} 
\sphinxAtStartPar
fluidi:
\begin{itemize}
\item {} 
\sphinxAtStartPar
in base all’equazione di stato:
\begin{itemize}
\item {} 
\sphinxAtStartPar
gas perfetti

\item {} 
\sphinxAtStartPar
gas reali

\item {} 
\sphinxAtStartPar
…

\end{itemize}

\item {} 
\sphinxAtStartPar
in base all’espressione degli sforzi:
\begin{itemize}
\item {} 
\sphinxAtStartPar
fluidi newtoniani

\item {} 
\sphinxAtStartPar
fluidi non\sphinxhyphen{}newtoniani

\end{itemize}

\end{itemize}

\end{itemize}

\sphinxstepscope

\begin{sphinxuseclass}{sd-container-fluid}
\begin{sphinxuseclass}{sd-sphinx-override}
\begin{sphinxuseclass}{sd-p-0}
\begin{sphinxuseclass}{sd-mt-2}
\begin{sphinxuseclass}{sd-mb-4}
\begin{sphinxuseclass}{sd-row}
\begin{sphinxuseclass}{sd-row-cols-2}
\begin{sphinxuseclass}{sd-gx-2}
\begin{sphinxuseclass}{sd-gy-1}
\begin{sphinxuseclass}{sd-col}
\begin{sphinxuseclass}{sd-d-flex-row}
\begin{sphinxuseclass}{sd-align-minor-center}
\begin{sphinxuseclass}{sd-container-fluid}
\begin{sphinxuseclass}{sd-sphinx-override}
\begin{sphinxuseclass}{sd-row}
\begin{sphinxuseclass}{sd-row-cols-2}
\begin{sphinxuseclass}{sd-row-cols-xs-2}
\begin{sphinxuseclass}{sd-row-cols-sm-3}
\begin{sphinxuseclass}{sd-row-cols-md-3}
\begin{sphinxuseclass}{sd-row-cols-lg-3}
\begin{sphinxuseclass}{sd-gx-3}
\begin{sphinxuseclass}{sd-gy-1}
\begin{sphinxuseclass}{sd-col}
\begin{sphinxuseclass}{sd-col-auto}
\begin{sphinxuseclass}{sd-d-flex-row}
\begin{sphinxuseclass}{sd-align-minor-center}
\sphinxAtStartPar
basics

\end{sphinxuseclass}
\end{sphinxuseclass}
\end{sphinxuseclass}
\end{sphinxuseclass}
\begin{sphinxuseclass}{sd-col}
\begin{sphinxuseclass}{sd-col-auto}
\begin{sphinxuseclass}{sd-d-flex-row}
\begin{sphinxuseclass}{sd-align-minor-center}
\sphinxAtStartPar
Dec 30, 2024

\end{sphinxuseclass}
\end{sphinxuseclass}
\end{sphinxuseclass}
\end{sphinxuseclass}
\begin{sphinxuseclass}{sd-col}
\begin{sphinxuseclass}{sd-col-auto}
\begin{sphinxuseclass}{sd-d-flex-row}
\begin{sphinxuseclass}{sd-align-minor-center}
\sphinxAtStartPar
0 min read

\end{sphinxuseclass}
\end{sphinxuseclass}
\end{sphinxuseclass}
\end{sphinxuseclass}
\end{sphinxuseclass}
\end{sphinxuseclass}
\end{sphinxuseclass}
\end{sphinxuseclass}
\end{sphinxuseclass}
\end{sphinxuseclass}
\end{sphinxuseclass}
\end{sphinxuseclass}
\end{sphinxuseclass}
\end{sphinxuseclass}
\end{sphinxuseclass}
\end{sphinxuseclass}
\end{sphinxuseclass}
\end{sphinxuseclass}
\end{sphinxuseclass}
\end{sphinxuseclass}
\end{sphinxuseclass}
\end{sphinxuseclass}
\end{sphinxuseclass}
\end{sphinxuseclass}
\end{sphinxuseclass}
\end{sphinxuseclass}

\section{Gas ideali}
\label{\detokenize{ch/ideal_gases:gas-ideali}}\label{\detokenize{ch/ideal_gases:classical-thermodynamics-ideal-gases}}\label{\detokenize{ch/ideal_gases::doc}}
\sphinxAtStartPar
\sphinxstylestrong{Legge di Boyle\sphinxhyphen{}Mariotte.} \(P V = \text{cost.}\) a \(T\) costante (trasformazione isoterma).

\sphinxAtStartPar
\sphinxstylestrong{Legge di Gay\sphinxhyphen{}Lussac I (o Charles).} \(V \propto T\) a \(P\) costante (trasformazione isobara).

\sphinxAtStartPar
\sphinxstylestrong{Legge di Gay\sphinxhyphen{}Lussac II.} \(P \propto T\) a \(V\) costante (trasformazione isocora).

\sphinxAtStartPar
\sphinxstylestrong{Legge dei gas ideali.} \(P V = n R T\)

\sphinxstepscope

\begin{sphinxuseclass}{sd-container-fluid}
\begin{sphinxuseclass}{sd-sphinx-override}
\begin{sphinxuseclass}{sd-p-0}
\begin{sphinxuseclass}{sd-mt-2}
\begin{sphinxuseclass}{sd-mb-4}
\begin{sphinxuseclass}{sd-row}
\begin{sphinxuseclass}{sd-row-cols-2}
\begin{sphinxuseclass}{sd-gx-2}
\begin{sphinxuseclass}{sd-gy-1}
\begin{sphinxuseclass}{sd-col}
\begin{sphinxuseclass}{sd-d-flex-row}
\begin{sphinxuseclass}{sd-align-minor-center}
\begin{sphinxuseclass}{sd-container-fluid}
\begin{sphinxuseclass}{sd-sphinx-override}
\begin{sphinxuseclass}{sd-row}
\begin{sphinxuseclass}{sd-row-cols-2}
\begin{sphinxuseclass}{sd-row-cols-xs-2}
\begin{sphinxuseclass}{sd-row-cols-sm-3}
\begin{sphinxuseclass}{sd-row-cols-md-3}
\begin{sphinxuseclass}{sd-row-cols-lg-3}
\begin{sphinxuseclass}{sd-gx-3}
\begin{sphinxuseclass}{sd-gy-1}
\begin{sphinxuseclass}{sd-col}
\begin{sphinxuseclass}{sd-col-auto}
\begin{sphinxuseclass}{sd-d-flex-row}
\begin{sphinxuseclass}{sd-align-minor-center}
\sphinxAtStartPar
basics

\end{sphinxuseclass}
\end{sphinxuseclass}
\end{sphinxuseclass}
\end{sphinxuseclass}
\begin{sphinxuseclass}{sd-col}
\begin{sphinxuseclass}{sd-col-auto}
\begin{sphinxuseclass}{sd-d-flex-row}
\begin{sphinxuseclass}{sd-align-minor-center}
\sphinxAtStartPar
Dec 30, 2024

\end{sphinxuseclass}
\end{sphinxuseclass}
\end{sphinxuseclass}
\end{sphinxuseclass}
\begin{sphinxuseclass}{sd-col}
\begin{sphinxuseclass}{sd-col-auto}
\begin{sphinxuseclass}{sd-d-flex-row}
\begin{sphinxuseclass}{sd-align-minor-center}
\sphinxAtStartPar
0 min read

\end{sphinxuseclass}
\end{sphinxuseclass}
\end{sphinxuseclass}
\end{sphinxuseclass}
\end{sphinxuseclass}
\end{sphinxuseclass}
\end{sphinxuseclass}
\end{sphinxuseclass}
\end{sphinxuseclass}
\end{sphinxuseclass}
\end{sphinxuseclass}
\end{sphinxuseclass}
\end{sphinxuseclass}
\end{sphinxuseclass}
\end{sphinxuseclass}
\end{sphinxuseclass}
\end{sphinxuseclass}
\end{sphinxuseclass}
\end{sphinxuseclass}
\end{sphinxuseclass}
\end{sphinxuseclass}
\end{sphinxuseclass}
\end{sphinxuseclass}
\end{sphinxuseclass}
\end{sphinxuseclass}
\end{sphinxuseclass}

\chapter{Trasformazioni termodinamiche}
\label{\detokenize{ch/thermodynamic_transformations:trasformazioni-termodinamiche}}\label{\detokenize{ch/thermodynamic_transformations:classical-thermodynamics-transformations}}\label{\detokenize{ch/thermodynamic_transformations::doc}}
\sphinxstepscope

\begin{sphinxuseclass}{sd-container-fluid}
\begin{sphinxuseclass}{sd-sphinx-override}
\begin{sphinxuseclass}{sd-p-0}
\begin{sphinxuseclass}{sd-mt-2}
\begin{sphinxuseclass}{sd-mb-4}
\begin{sphinxuseclass}{sd-row}
\begin{sphinxuseclass}{sd-row-cols-2}
\begin{sphinxuseclass}{sd-gx-2}
\begin{sphinxuseclass}{sd-gy-1}
\begin{sphinxuseclass}{sd-col}
\begin{sphinxuseclass}{sd-d-flex-row}
\begin{sphinxuseclass}{sd-align-minor-center}
\begin{sphinxuseclass}{sd-container-fluid}
\begin{sphinxuseclass}{sd-sphinx-override}
\begin{sphinxuseclass}{sd-row}
\begin{sphinxuseclass}{sd-row-cols-2}
\begin{sphinxuseclass}{sd-row-cols-xs-2}
\begin{sphinxuseclass}{sd-row-cols-sm-3}
\begin{sphinxuseclass}{sd-row-cols-md-3}
\begin{sphinxuseclass}{sd-row-cols-lg-3}
\begin{sphinxuseclass}{sd-gx-3}
\begin{sphinxuseclass}{sd-gy-1}
\begin{sphinxuseclass}{sd-col}
\begin{sphinxuseclass}{sd-col-auto}
\begin{sphinxuseclass}{sd-d-flex-row}
\begin{sphinxuseclass}{sd-align-minor-center}
\sphinxAtStartPar
basics

\end{sphinxuseclass}
\end{sphinxuseclass}
\end{sphinxuseclass}
\end{sphinxuseclass}
\begin{sphinxuseclass}{sd-col}
\begin{sphinxuseclass}{sd-col-auto}
\begin{sphinxuseclass}{sd-d-flex-row}
\begin{sphinxuseclass}{sd-align-minor-center}
\sphinxAtStartPar
Dec 30, 2024

\end{sphinxuseclass}
\end{sphinxuseclass}
\end{sphinxuseclass}
\end{sphinxuseclass}
\begin{sphinxuseclass}{sd-col}
\begin{sphinxuseclass}{sd-col-auto}
\begin{sphinxuseclass}{sd-d-flex-row}
\begin{sphinxuseclass}{sd-align-minor-center}
\sphinxAtStartPar
1 min read

\end{sphinxuseclass}
\end{sphinxuseclass}
\end{sphinxuseclass}
\end{sphinxuseclass}
\end{sphinxuseclass}
\end{sphinxuseclass}
\end{sphinxuseclass}
\end{sphinxuseclass}
\end{sphinxuseclass}
\end{sphinxuseclass}
\end{sphinxuseclass}
\end{sphinxuseclass}
\end{sphinxuseclass}
\end{sphinxuseclass}
\end{sphinxuseclass}
\end{sphinxuseclass}
\end{sphinxuseclass}
\end{sphinxuseclass}
\end{sphinxuseclass}
\end{sphinxuseclass}
\end{sphinxuseclass}
\end{sphinxuseclass}
\end{sphinxuseclass}
\end{sphinxuseclass}
\end{sphinxuseclass}
\end{sphinxuseclass}

\chapter{Cicli termodinamici e macchine termiche}
\label{\detokenize{ch/heat_engines:cicli-termodinamici-e-macchine-termiche}}\label{\detokenize{ch/heat_engines:classical-thermodynamics-heat-engines}}\label{\detokenize{ch/heat_engines::doc}}
\sphinxAtStartPar
Le macchine termiche sono sistemi che sfruttano scambi di calore per produrre lavoro (\sphinxstylestrong{macchine dirette}, come i motori a combustione) o lavoro per scambiare calore da sistemi freddi a sistemi più caldi (\sphinxstylestrong{macchine inverse}, come i frigoriferi).

\sphinxAtStartPar
Di solito, le macchine termiche sfruttano un fluido di lavoro. Le macchine a fluido possono essere un sistema aperto (es. motori aeronautici) o chiuso (circuiti delle centrali elettriche e di frigoriferi), o un sistema che è aperto solo in alcune fasi (es. nei motori alternativi, la camera di combustione è un sistema aperto solo durante le fasi di aspirazione e scarico, se si trascurano le perdite).

\sphinxAtStartPar
\sphinxstylestrong{Macchine ideali.} \sphinxstyleemphasis{Macchina ideale di Carnot; efficienza massima; enunciati di Planck e Kelvin del secondo principio della termodinamica.}

\sphinxAtStartPar
\sphinxstylestrong{Macchine reali.} \sphinxstyleemphasis{Cicli ideali: Otto, Diesel, Rankine, Joule\sphinxhyphen{}Brayton,…;}

\sphinxstepscope

\begin{sphinxuseclass}{sd-container-fluid}
\begin{sphinxuseclass}{sd-sphinx-override}
\begin{sphinxuseclass}{sd-p-0}
\begin{sphinxuseclass}{sd-mt-2}
\begin{sphinxuseclass}{sd-mb-4}
\begin{sphinxuseclass}{sd-row}
\begin{sphinxuseclass}{sd-row-cols-2}
\begin{sphinxuseclass}{sd-gx-2}
\begin{sphinxuseclass}{sd-gy-1}
\begin{sphinxuseclass}{sd-col}
\begin{sphinxuseclass}{sd-d-flex-row}
\begin{sphinxuseclass}{sd-align-minor-center}
\begin{sphinxuseclass}{sd-container-fluid}
\begin{sphinxuseclass}{sd-sphinx-override}
\begin{sphinxuseclass}{sd-row}
\begin{sphinxuseclass}{sd-row-cols-2}
\begin{sphinxuseclass}{sd-row-cols-xs-2}
\begin{sphinxuseclass}{sd-row-cols-sm-3}
\begin{sphinxuseclass}{sd-row-cols-md-3}
\begin{sphinxuseclass}{sd-row-cols-lg-3}
\begin{sphinxuseclass}{sd-gx-3}
\begin{sphinxuseclass}{sd-gy-1}
\begin{sphinxuseclass}{sd-col}
\begin{sphinxuseclass}{sd-col-auto}
\begin{sphinxuseclass}{sd-d-flex-row}
\begin{sphinxuseclass}{sd-align-minor-center}
\sphinxAtStartPar
basics

\end{sphinxuseclass}
\end{sphinxuseclass}
\end{sphinxuseclass}
\end{sphinxuseclass}
\begin{sphinxuseclass}{sd-col}
\begin{sphinxuseclass}{sd-col-auto}
\begin{sphinxuseclass}{sd-d-flex-row}
\begin{sphinxuseclass}{sd-align-minor-center}
\sphinxAtStartPar
Dec 30, 2024

\end{sphinxuseclass}
\end{sphinxuseclass}
\end{sphinxuseclass}
\end{sphinxuseclass}
\begin{sphinxuseclass}{sd-col}
\begin{sphinxuseclass}{sd-col-auto}
\begin{sphinxuseclass}{sd-d-flex-row}
\begin{sphinxuseclass}{sd-align-minor-center}
\sphinxAtStartPar
0 min read

\end{sphinxuseclass}
\end{sphinxuseclass}
\end{sphinxuseclass}
\end{sphinxuseclass}
\end{sphinxuseclass}
\end{sphinxuseclass}
\end{sphinxuseclass}
\end{sphinxuseclass}
\end{sphinxuseclass}
\end{sphinxuseclass}
\end{sphinxuseclass}
\end{sphinxuseclass}
\end{sphinxuseclass}
\end{sphinxuseclass}
\end{sphinxuseclass}
\end{sphinxuseclass}
\end{sphinxuseclass}
\end{sphinxuseclass}
\end{sphinxuseclass}
\end{sphinxuseclass}
\end{sphinxuseclass}
\end{sphinxuseclass}
\end{sphinxuseclass}
\end{sphinxuseclass}
\end{sphinxuseclass}
\end{sphinxuseclass}

\chapter{Meccanismi di trasmissione del calore}
\label{\detokenize{ch/heat_transmission:meccanismi-di-trasmissione-del-calore}}\label{\detokenize{ch/heat_transmission:classical-thermodynamics-heat-transmission}}\label{\detokenize{ch/heat_transmission::doc}}






\renewcommand{\indexname}{Index}
\printindex
\end{document}